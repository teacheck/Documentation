\chapter*{\abstract}

\setcounter{page}{1}
“Teacheck surgió de la necesidad de un sistema capaz de detectar el deterioro en el rendimiento de un alumno en una universidad de forma precisa y eficaz. El siguiente informe  contempla el análisis, definición y visualización de la información necesaria para la correcta aplicación de este concepto. La solución planteada agrupa la enseñanza automática requerida para el análisis, el servicio web para su visualización y la estructura de datos que será clave para mantener el flujo de datos para que el análisis estadístico de los resultados sea lo más preciso y eficaz posible. A continuación, se recoge tanto el informe de los aspectos y herramientas técnicas utilizadas para el desarrollo de la ampliación, como la valoración de las competencias de la aplicación desarrolladas durante esas tareas.”
	
