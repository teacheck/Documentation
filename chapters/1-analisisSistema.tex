\chapter{Análisis del Sistema}
% \thispagestyle{empty}
\section{Objetivos}
Teacheck es una aplicación para realizar el seguimiento del alumnado de forma automática y precisa. El principal objetivo de esta, es reducir en gran parte la tasa de abandono, repetidores y suspensos para así aumentar la tasa de rendimiento y graduación de una universidad. También se van a poder analizar diferentes problemas que puedan surgir a nivel de toda una clase gracias a los diferentes datos que almacenaremos. Con todo esto, conseguiremos aumentar el rendimiento y eficiencia de las clases y en consecuencia, el de la universidad en general, afectando positivamente tanto en su prestigio como en su valor como institución, lo que atraerá a nuevos alumnos y empresas.  
Al tratarse de una aplicación para universidades, estas serán las que compren nuestros servicios tanto para un curso como para todo un grado. Por tanto,
todo beneficio económico será el relacionado con nuevas matrículas de los alumnos, que, al fin y al cabo, se han previsto que consigan gracias al prestigio que pueda otorgar la aplicación.
Por otro lado, Teacheck se encargará de conseguir el formato necesario para alimentar el sistema, así pues, la universidad únicamente deberá dar acceso a la fuente de dichos datos, minimizando así los costes de implementación de la aplicación por parte de las universidades.
\section{Definición de la empresa y del servicio}
Teacheck ofrece a estos centros una aplicación web con la cual mediante sus servicios es capaz de realizar un seguimiento de sus alumnos, gracias a tecnología innovadora como la inteligencia artificial.
Es una empresa destinada a instituciones o universidades tanto públicas como concertadas que busquen realizar una monitorización automática y precisa de su alumnado. Ofrece un servicio con el objetivo de facilitar un seguimiento en beneficio tanto del alumnado principalmente como del profesor. Con esto pretendemos resolver el problema que actualmente existe con el alto porcentaje de repetidores y suspensos. Creemos que la mayoría de estos casos suceden por falta de responsabilidad y desconocimiento de cuando el alumno va mal o su rendimiento no es el adecuado, y gran parte de estos casos son evitables. Y para dar solución a este problema, el primer paso es el interés del profesorado en intentar revertir esta situación y tratar de alertar al alumno de su situación, que, aunque él ya sea consciente de ello, en la mayoría de los casos, el hecho de recibir un aviso o consejo de forma adecuada, esto es, teniendo en cuenta cual es la manera correcta de decir y plantear los problemas, puede hacerle cambiar de actitud. Por lo tanto, para conseguir esto es necesario el seguimiento del alumno en cuestión y tener presentes sus notas, asistencia, motivación y sus entregables entre otros. Pero aquí se nos plantea otro problema, y es que hacer el seguimiento de un alumno es fácil, pero no es lo mismo con diez, veinte o treinta de ellos. Entonces se complican las cosas ya que el profesor o tutor en cuestión no podrá cumplir con el seguimiento de todos y los avisos no podrán llegar a tiempo. 
\subsection{Organigrama de funcionamiento}
Como podemos ver en el organigrama de abajo, la empresa está formada por 4 departamentos o unidades que se basan en:
\begin{itemize}
\item{El producto, esto es,el desarrollo de toda la aplicación en general. Podríamos dividir este departamento en diferentes secciones o roles como podrían ser: El análisis y diseño, el desarrollo del software y el análisis estadístico y la inteligencia artifical del producto.}
\item{Monitorizar y gestionar el capital de la empresa así como las inversiones.}
\item{La gestión los empleados y lo relacionado con nuevas incorporaciones al equipo y su bienestar.}
\item{Por último ,contactar nuevas universidades y promoción de la aplicación.}
\end{itemize}
\subparagraph{}
\irudia{template/figs/organigrama-empresa.jpg}{1}{Organigrama de la empresa}
\subsection{Ventajas}
Como hemos mencionado anteriormente, gracias al seguimiento personalizado que ofrece Teacheck, una universidad podrá organizar de una forma fácil y sencilla toda información relacionada con el rendimiento de un alumno o clase en general.Así el  personal docente y los alumnos tendrán una forma de detectar el motivo de una posible degradación en el ritmo de trabajo de un alumno o profesor.
Además, como ha sido mencionado anteriormente, gracias a este sistema, se tendrán en cuenta toda la información de todos los alumnos de una clase para saber si una asignatura en concreto es implantada de forma correcta.
\subsection{Desventajas}
La participación del alumnado no es necesaria para el correcto funcionamiento del sistema, pero su eficiencia y exactitud incrementa exponencialmente si dispone de la información que un alumno puede llegar a ingresar. Es por esto que, aún siendo opcional, los alumnos se puedan sentir presionados o en cierta forma controlados por la universidad a la hora de tener que ingresar datos como puedan ser las horas de estudio fuera de clase, la asistencia o la motivación en el aula.
El personal docente del centro deberá recoger tanto las notas como la asistencia diaria de sus alumnos y calificar cada clase que imparta, ya que después, las observaciones hechas durante esa clase, serán claves para saber el estado de sus alumnos, aumentando así las responsabilidades de un profesor con una asignatura.
\subsection{Elementos diferenciadores}
Teacheck se situa en la categoría de aplicaciones que intentan prevenir un deterioro en el rendimiento académico de los alumnos. Dentro de esta categoría ya existe una institución que realiza este tipo de actividad. Pero Teacheck ofrece algo más que solamente el análisis de los datos proporcionados por el profesor, también ofrece la posibilidad de que un alumno pueda proporcionar datos los cuales permitirá a la aplicación ser más precisa y eficaz.
Ya que de esta manera, los datos introducidos por los profesores nos dirán en qué está fallando el alumno y los datos proporcionados por el alumno, cómo solucionarlo.
\section{Arquitectura del sistema}
\irudia{template/figs/arch.png}{1}{Arquitectura del Sistema}
\section{Diseño centrado en el usuario}
Una buena aplicación necesita un buen diseño, pero para que sea bueno no es suficiente hacerlo bonito, también es necesario hacerlo útil. Para asegurarnos de esto, hemos seguido lo que se llama Diseño centrado en el usuario. DCU es una metodología enfocada a las necesidades reales del usuario.
\subsection{Contexto de uso}
Primero debemos analizar los diferentes tipos de usuario que la aplicación tiene y como interactúan ellos con la aplicación.

Algunas partes de la aplicación pueden, y probablemente serán complejas, pero eso no debería ser un problema. La complejidad no siempre significa dificultad, a veces las tareas complejas se pueden realizar fácilmente. Tiene que ser una herramienta poderosa, que nos aporte mucho, pero a la vez, que sea fácil de utilizar.
\subparagraph{Entorno}
Esta aplicación está diseñada para Universidades y son los los integrantes de esa universidad los que la utilizarán. Debe ser una herramienta útil que nos proporcione la información necesaria para conseguir los objetivos. Por lo tanto, la eficiencia es mucho más importante que la estética.
\subsection{Usuarios}
\begin{itemize}
\item \textbf{Alumnos:}
Es una aplicación en la que los alumnos no pasarán mucho tiempo ya que no tendrá muchas funcionalidades. Por una lado, los alumnos podrán ver su perfil (foto, datos personales, estadísticas generales,…). Por otro lado,  podrán ver el estado en el que están, es decir, estadísticas e información más ecisa tales como estado del curso, sus notas, asistencia, entregables, etc.  Y por último, podrán realizar las encuestas semanales. Estas encuestas son rápidas y sencillas pero que serán muy útiles a la hora de analizar a los alumnos.

Cada año, nuevos alumnos comenzarán a utilizar la aplicación y otros que ya la utilizaban dejarán de hacerlo al terminar sus estudios. Por lo tanto, tiene que ser una aplicación fácil de aprender y de utilizar.
\item \textbf{Profesores:}
Probablemente, son los usuarios que más utilizarán la aplicación. Estos, tendrán disponibles tres apartados. En el primero, podrán ver el estado de todos los alumnos de cada curso donde da clase. En el segundo, tendrá el apartado de alarmas donde verá que alumnos suyos están en peligro o necesitan un toque de atención. Y por último, tendrá el apartado de las estadísticas generales, es decir, se le mostraran los resultados generales de las encuestas a nivel de clase.
\item \textbf{Coordinadores del curso:}
Los coordinadores del curso, al igual que los profesores, también pasarán tiempo con esta aplicación. Y al igual que los profesores, podrán ver el estado de los alumnos de el curso o los cursos que coordina, el apartado de las alarmas y los resultados generales de las encuestas semanales. Pero a diferencia de los profesores, las alarmas que se le notifiquen no serán solo a nivel de alumno. También se le avisarán de los alarmas a nivel de clase.
\end{itemize}
\subsection{Requisitos}
\subsubsection{Requisitos de negocio}
\paragraph{\textsc {General}}
\begin{itemize}
\item {La aplicación permitirá hacer diferentes cosas al usuario según el rol asignado.}
\item {La aplicación debe estar protegida por un login.}
\item {La aplicación guardará los datos de manera protegida ya que maneja datos delicados.}
\end{itemize}
\paragraph{\textsc {Alumnos}}
\begin{itemize}
\item {La aplicación tiene que ser capaz de visualizar estadísticas de los alumnos tales como el estado, notas, asistencia y entregables.}
\item {La aplicación debe ser capaz de mostrar el perfil al usuario con su foto y datos personales.}
\item {La aplicación permitirá rellenar a los alumnos encuestas semanales.}
\item {La aplicación hará análisis periódicos de los alumnos con las fechas definidas anteriormente.}
\end{itemize}
\paragraph{\textsc {Profesores}}
\begin{itemize}
\item {La aplicación permitirá a los profesores ver el el estado de los alumnos de las clases donde imparte clases.}
\item {La aplicación permitirá a los profesores ver el apartado de alarmas donde verá todos los alumnos a los que le ha saltado la alarma.}
\item {La aplicación mostrará a los profesores los resultados generales de las encuestas por cada asignatura que enseña.}
\item {La aplicación deberá enviar dichas alarmas a los profesores vía email.}
\end{itemize}
\paragraph{\textsc {Coordinadores del curso}}
\begin{itemize}
\item {La aplicación permitirá a los coordinadores ver el el estado de los alumnos del curso que coordina.}
\item {La aplicación permitirá a los coordinadores ver el apartado de alarmas donde verá todos los alumnos a los que le ha saltado la alarma y todas las alarmas a nivel de clase.}
\item {La aplicación mostrará a los coordinadores los resultados generales de las encuestas por cada curso que coordina.}
\item {La aplicación deberá enviar dichas alarmas a los coordinadores vía email.}
\end{itemize}
\subsubsection{Requisitos de usuario}
\paragraph{\textsc {Alumnos}}
\begin{itemize}
\item {Los alumnos tienen que ser capaces de ver su estado del curso mediante tablas, listas o dasboards.}
\item {Los alumnos tienen que ser capaces de acceder a su perfil mediante un simple botón situado a la izquierda. }
\item {Los alumnos tienen que ser capaces de rellenar encuestas semanales mediante formularios.}
\end{itemize}
\paragraph{\textsc {Profesores}}
\begin{itemize}
\item {Los profesores tienen que ser capaces de ver el estado de sus alumnos a través de listas, tablas de datos o dashboards.}
\item {Los profesores tienen que ser capaces de ver las alarmas de dichos alumnos mediante listas.}
\item {Los profesores tienen que ser capaces de ver los resultados de las encuestas mediante tablas o gráficos.}
\item {Los profesores tienen que ser capaces de acceder a su perfil mediante un simple botón situado a la izquierda.}
\end{itemize}
\paragraph{\textsc {Coordinadores del curso}}
\begin{itemize}
\item {Los coordinadores tienen que ser capaces de ver el estado de sus alumnos a través de listas, tablas de datos o dashboards.}
\item {Los coordinadores tienen que ser capaces de ver las alarmas de dichos alumnos mediante listas.}
\item {Los coordinadores tienen que ser capaces de ver los resultados de las encuestas mediante tablas o gráficos.}
\item {Los coordinadores tienen que ser capaces de acceder a su perfil mediante un simple botón situado a la izquierda.}
\end{itemize}
\subsubsection{Requisitos funcionales}
\begin{itemize}
\item {API de Highcharts. Esto se necesitará para crear gráficos con altos detalles de visualización de datos.}
\item {API’s de Vertx. Se utilizarán para la construcción del sistema. }
\item {La aplicación debe desarrollarse en el lenguaje de programación Java.}
\item {Se va a utilizar Docker para crear una imagen de la aplicación para implementarla en un servidor remoto.}
\item {La aplicación tiene que estar desarrollada en una plataforma multilenguaje.}
\end{itemize}
\subsubsection{Requisitos de calidad de servicio}
\begin{itemize}
\item {La aplicación estará disponible 24/7.}
\item {La aplicación debe ser escalable, es decir, al introducir nuevas asignaturas no debe influir en el modelo de la arquitectura.}
\item {El sistema debe ser tolerante a fallos, incluso si ocurre un problema de hardware, el sistema debe seguir funcionando como se espera.}
\item {La seguridad de la aplicación debe incluir autorización, autenticación, seguridad de acceso a datos y acceso seguro para el sistema de implementación, así como su administración.}
\item {El sistema debe ser mantenible, monitoreado y actualizado cuando sea necesario.}
\end{itemize}
\section{Base de datos}
A la hora de diseñar la base de datos hemos intentado construir un modelo que permita flexibilidad para poder adaptarnos a los distintos clientes y a sus requisitos. Para ello tenemos el siguiente modelo entidad relación.
%\irudia{template/figs/db-Page-1.png}{2}{Base de datos}
Se puede decir que el núcleo se encuentra en la relación entre curso, alumno y asignatura, creando una relación llamada matrícula donde se registrará al alumno con sus respectivas asignatura y curso. Con esto conseguimos tener una mayor flexibilidad a la hora de posibles casos individuales que puedan afectar a un sistema más fijo sin opción a modificaciones. Luego tenemos al profesor que podrá tanto impartir una o más asignaturas y a la vez coordinar un curso. También hemos definido una entidad para poder llevar los registros históricos del feedback semanal entregado por el alumno. Por último la entidad asignatura tendrá tantos exámenes, notas de comportamiento y entregables como sean requeridos. A continuación en el modelo relacional veremos qué atributos completan cada entidad.
%\irudia{template/figs/db-Page-2.png}{3}{Modelo relacional}
La relación matrícula creará una nueva entidad con las claves primarias de las entidades que crean dicha relación. Del curso se guardará su nombre, el coordinador y una breve descripción del mismo. El alumno tendrá datos personales para identificarlo y ‘repetidor’ que será un booleano para saber si no tiene las asignaturas estándar de un curso o puede que esté matriculado en más de uno. Luego la asignatura guardará los datos necesarios para identificarla, peso correspondiente en el curso y se completa tanto con los entregables, exámenes y notas de comportamiento que guardarán la fecha de cuando se registran, la competencia a la que pertenecen, y datos de identificación. Por último, para registrar los feedback de alumnado nos basta con guardar la fecha y los tres atributos recogidos en el feedback. 


