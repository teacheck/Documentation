\chapter{Introducción}
% \thispagestyle{empty}
\section{Problema}

Lo primero, comenzaremos con hacer un análisis de la problemática que hemos detectado, haciendo énfasis en los aspectos más importantes del mismo. También hablaremos de la solución que se ha intentado dar a dicho problema, y cómo esta solución da pié a otros problemas que no le permiten solucionar ni lograr los objetivos que se plantea.
\begin{itemize}
\item{\textbf{Problema: }}Alto índice de repetidores y abandonos en las universidades.
\end{itemize}

A día de hoy, uno de cada tres alumnos abandona la carrera universitaria que comienza a cursar. De entre ellos, una tercera parte cambia de carrera y la otra abandona la universidad. Los datos son alarmantes y no muchas universidades toman medidas al respecto. 

Las razones por la que esto sucede son varias. Para empezar, la desinformación por parte de los alumnos a la hora de elegir la carrera da como resultado que muchos de ellos no tomen la decisión correcta y que al cabo de unas semanas o meses de comenzar no se sientan cómodos con la elección realizada. Por otra parte, la tasa de graduación ronda entre el 64\% y el 83\% lo cual indica que el número de asignaturas suspendidas y repetidas es elevada y esto lleva a muchos alumnos a no querer continuar con la carrera.\cite{elMundoProblema},\cite{stecylProblema}

Teniendo esto en cuenta, nos hemos centrado en analizar la tasa de repetidores. Hay casos en los que el alumno en cuestión no consigue el aprobado debido a una falta de capacidad y es algo a tener en cuenta pero no es la razón mayoritaria. La falta de responsabilidad, motivación y atención en clase por parte del alumno son los verdaderos problemas del bajo índice de graduación universitaria. Y es que otros problemas pueden llegar a afectar a dicho índice, pero estos son la verdadera clave. La razón por la que estos problemas suceden son más complicados de analizar y predecir, y se necesita de un conocimiento psicológico y análisis profundo del problema en cuestión para poder llegar a una conclusión, pero sí que podemos asegurar, tal y como hemos y han observado muchos alumnos universitarios, que en parte se dan a unos malos hábitos de estudio y responsabilidad.
\begin{itemize}
\item{\textbf{Solución: }}Seguimiento del alumno por parte del profesorado.
\end{itemize}

Para poder combatir las tasas antes mencionadas muchas universidades optan por hacer un seguimiento de su alumnado. Generalmente recogen los datos que los alumnos van generando a lo largo del curso como pueden ser las notas, entregables, asistencia, entre otros. Básicamente analizan los datos y cuando consideran que un alumno no lleva una buena dinámica o sus resultados no son los correctos, lo alertan y tratan de revertir su situación.

Pero aun haciendo esto, la realidad es que las tasas no sufren grandes cambios debido a los diferentes problemas que se detallan a continuación.
\begin{itemize}
\item{\textbf{Nuevos problemas}}
\paragraph{}
\textbf{Demasiados alumnos:}
\paragraph{}
Hacer el seguimiento de un alumno puede llegar a ser posible, incluso conseguir revertir su situación, pero porque no cambian las tasas entonces? Esto se debe a que el número de alumnos es demasiado elevado, y son los profesores normalmente los que se ocupan de dicho seguimiento.Esto implica que cada profesor se tiene que ocupar de sus más de 200 300 o incluso 500 alumnos de hasta 4 carreras distintas. Y no solo eso, si no que a su vez debería coordinarse con los demás profesores de cada alumno para poder determinar con exactitud cuando un alumno va mal. Por lo tanto, la carga de trabajo que se le añade a un profesor al asignarle la tarea de hacer el seguimiento es inmensa y eso hace que el seguimiento no pueda efectuarse adecuadamente.

Además creemos que el seguimiento tiene dos aspectos importantes. La primera, que consiste en poder detectar a tiempo el deterioro del rendimiento del alumno, y la segunda, el comunicar de forma efectiva dicho problema al alumno para conseguir revertir la situación. Pero para esto último no se suele contar con el mismo protagonista, esto es, el alumno, lo cual es un error que contemplamos a continuación.
\paragraph{}
\textbf{No se cuenta con la opinión del alumno:}
\paragraph{}
Tal y como se acaba de mencionar, no contar con la opinión o información que un alumno pueda aportar a dicho seguimiento es un error, ya que para conseguir revertir su situación no solamente se trata de detectar el problema a tiempo, si no que también se deben conocer los detonantes que han derivado en dicho problema. Por ello, creemos que obviar la información que un alumno pueda aportar es algo que no se debe hacer.
\end{itemize}

\section{Objetivos}

Visto y analizado el problema, el objetivo que se plantea es poder ofrecer un producto que cumpla con los siguientes objetivos.
\begin{itemize}
\item{\textbf{Analizar:}}
\end{itemize}
\paragraph{}
El primer objetivo es conseguir un análisis del alumno preciso y eficaz. Debido a que el problema se debería detectar a tiempo, también es esencial que se analize la situación del alumno constantemente. También, tal y como se ha mencionado en el análisis del problema, saber cuales son los detonantes es algo necesario para que el responsable que trate de revertir la situación del alumno cuente con esa información que consideramos necesario para que logre su objetivo. También queremos llegar más allá y detectar posibles bajos rendimientos a nivel de clase, esto es, poder ver si existe algún patrón negativo a nivel de clase y cuál podría ser el detonante. 
\begin{itemize}
\item{\textbf{Automatizar:}}
\end{itemize}
Otro de los puntos importantes a la hora de analizar el problema era que a los profesores les suponía demasiada carga de trabajo el tener que realizar el seguimiento. Por ello, otra de las claves del producto será lograr una automatización que libre a los profesores de dicha carga.
\begin{itemize}
\item{\textbf{Desplegar información:}}
\end{itemize}
A su vez, queremos crear una plataforma en la que los profesores puedan ver los resultados de todo el seguimiento.
\begin{itemize}
\item{\textbf{El alumno:}}
\end{itemize}
Tal y como se ha mencionado, consideramos que la información que un alumno puede proveer es esencial para poder tanto realizar un seguimiento efectivo como para poder conocer los detonantes de dichos problemas. Por ello, será importante que nuestra plataforma cuente con un sistema donde el alumno pueda aportar información, que más tarde se utilizará para realizar el seguimiento.
\begin{itemize}
\item{\textbf{El producto:}}
\end{itemize}
También mencionar que nuestro producto debe de ser escalable y seguro. 
\begin{itemize}
\item{\textbf{Objetivos cuantitativos:}}
\end{itemize}
\paragraph{}
Los objetivos que Teacheck quiere conseguir implantando su sistema en la universidad, como antes se ha mencionado, es incrementar o disminuir las tasas de ese centro hasta llegar al porcentaje que deseamos. Por ello los porcentajes mínimos a los que nos gustaría llegar son los siguientes:
\begin{enumerate}
\item\textbf{Incrementar}\\
\textbf{Éxito: } De los créditos a conseguir, cuantos se aprueban\\
(rango actual) - (rango que se quiere conseguir aumentar)\\
50\%-60\% -> 17\%\\
60\%-70\% -> 15\%\\
70\%-80\% -> 10\%\\
+80\% -> 7\%\\
\\
\textbf{Rendimiento: } De los créditos ordinarios a conseguir, cuantos se aprueban\\
50\%-60\% -> 17\%\\
60\%-70\% -> 15\%\\
70\%-80\% -> 10\%\\
+80\% -> 7\%\\
\\
\textbf{Graduación: } Cuántos alumnos acaban el curso en el tiempo previsto por la universidad\\
30\%-40\% -> 12\%\\
40\%-50\% -> 9\%\\
+60\% -> 7\%\\

\item\textbf{Disminuir}\\
\textbf{Abandono: } Cuántos alumnos abandonan el curso\\
40\%-30\% -> 14\%\\
30\%-20\% -> 11\%\\
-20\% -> 5\%\\
\\
\textbf{Repetidores: } De los alumnos que están cursando un año, cuantos no pasan de curso\\
40\%-30\% -> 14\%
30\%-20\% -> 11\%
-20\% -> 5\%
\\
\textbf{Suspensos: } De los exámenes o puntos de control hechos, cuántos son suspendidos\\
40\%-30\% -> 13\%\\
30\%-20\% -> 7\%\\
-20\% -> 5\%\\
\end{enumerate}

\section{Mercado}

\paragraph{}
La aplicación estará dirigida a universidades que busquen hacer un
seguimiento detallado de su alumnado. El usuario final del sistema,
por tanto, serán tanto los diferentes grupos de profesores o
coordinadores como los alumnos que se encuentren en la institución. El
ámbito geográfico que pretendemos abarcar es el de nivel
nacional. Antes de pensar en nuestra solución hemos analizado diferentes aplicaciones
y las hemos dividido en distintas categorías:

\begin{itemize}
\item Aplicaciones con IA que ayudan a los alumnos a aprender de una
  manera más eficiente y efectiva.
\begin{itemize}
\item \textbf{Easy learning:} Kidaptive es una plataforma de enseñanza
  adaptativa que impulsa una variedad de dominios de aprendizaje que
  incluyen dos aplicaciones creadas por Kidaptive para el aprendizaje
  temprano. Osmo es un juego interactivo que combina aprendizaje
  online y experiencial.
\end{itemize}
\item Aplicaciones que ayudan a los profesores en la enseñanza con la
  ayuda de la IA.
\begin{itemize}
\item \textbf{Contenido:} Los proveedores de contenido premium
  utilizan cada vez más el aprendizaje automático para ofrecer la
  siguiente mejor lección. Por ejemplo, startups como Content
  Technologies Inc. hacen uso de machine learning para automatizar su
  producción y automatización de procesos de negocio, diseño
  instruccional y soluciones de contenido y el proceso de enseñanza.
\end{itemize}
\item Aplicaciones sin IA que automatizan las actividades de monitoreo
  del profesor en respecto al
  alumnado.\cite{appsEvaluacionEstudiantes}
\begin{itemize}
\item \textbf{Additio:} Se trata de una herramienta versátil con
  muchas funcionalidades al alrededor del mundo educativo, entre ellas
  la capacidad de llevar un registro de notas de los estudiantes de
  forma muy visual, intuitiva y práctica.
\item \textbf{TeacherKit:} Permite crear diferentes clases, cada una
  con sus alumnos y un sinfín de opciones para cada una de
  ellas. TeacherKit ayuda a llevar un registro de notas y también de
  asistencias y de comportamiento, con la posibilidad de exportar
  todos los datos para gestionarlos por su cuenta.
\end{itemize}
\item Aplicaciones con IA que monitorean el rendimiento de los alumnos
  y sacan alarmas según los diferentes objetivos:
\begin{itemize}
\item Aplicaciones que tienen como objetivo prevenir el abandono de
  alumnos en respecto a su carrera.
\begin{itemize}
\item \textbf{Universidad de Derby:} donde se implementó un sistema de
  monitoreo de la deserción estudiantil que utiliza los datos para
  predecir qué estudiantes tienen riesgo de dejar sus estudios,
  permitiéndole a la institución intervenir antes de que ello
  suceda.\cite{riesgoDejarEstudios}
\end{itemize}
\item Aplicaciones que previenen el deterioro del rendimiento del
  alumnado con el fin de revertir la mala situación para obtener
  mejores resultados.
\begin{itemize}
\item \textbf{Universidad Internacional de la Rioja:} Un equipo de
  expertos de la Universidad Internacional de La Rioja trabaja en un
  proyecto piloto para, gracias al uso y aplicación de la Inteligencia
  Artificial (IA), poder medir ,mediante algoritmos, dicho
  rendimiento. Se analiza el comportamiento del alumno en la
  plataforma, su participación en los foros, su interacción con el
  material de estudio, las calificaciones intermedias obtenidas en la
  evaluación continua… Al poder compararse con los históricos de
  estudiantes anteriores, se observa si existe un
  patrón.\cite{rendimientoAlumnos}	
\end{itemize}
\end{itemize}
\end{itemize}

\section{Solución}

\paragraph{}
Una vez analizado el mercado y para cumplir con los objetivos definidos, se planteó Teacheck. Teacheck contará con distintas funcionalidades para cumplir con lo propuesto.
\begin{itemize}
\item{\textbf{Aplicación web: }}
\end{itemize}
La aplicación web contará con un sistema distribuido que será el responsable de proveer servicios básicos a la aplicación o puede que algún servicio adicional acordado con el cliente, se implemente con el sistema. Dicho sistema estará dividido en varios microservicios que ofrecen recursos esenciales a la aplicación principal. Servicios como, acceso a datos, el proveedor de datos de la Universidad, el sistema de IA Machine Learning y como dicho antes si es de interés del cliente, se podrá amoldar a algún sistema que existe y que ya provee todos los datos de los alumnos.\\
\\
\textbf{Roles:}
\paragraph{}
La aplicación dispone de dos roles distintos:
\begin{enumerate}
\item\textbf{Alumno: }podrá ver su seguimiento a diario y además, proveer datos al sistema de la aplicación ya que se le pedirá rellenar una encuesta semanalmente. Dicha encuesta preguntará detalles sobre el estado actual del alumno y su nivel de satisfacción en relación a las actividades lectivas.
\item\textbf{Profesor: }podrá monitorizar sus alumnos y cuando sea el caso recibirá alarmas de alumnos que se han detectado con bajo rendimiento o que necesiten atención.
\end{enumerate}

\begin{itemize}
\item{\textbf{Machine learning: }}
\end{itemize}
\\
\textbf{Análisis - Inteligencia artificial:}
\paragraph{}
Teacheck, es un sistema de monitoramento que ofrecerá periódicamente análisis sobre la situación de cada alumno registrado en la universidad. Contará con el sistema de Machine Learning (Inteligencia Artificial)  para efectuar los análisis. Estos escaneos de información son específicamente predicciones en base a un modelo que ya dispone el sistema con la información inicial proporcionada por el cliente.

Cada análisis traerá consigo resultados sobre cada alumno en respecto a su rendimiento en el curso correspondiente. Contaremos tanto con los datos recogidos por el profesor como con los que el alumno pueda proporcionar. Por parte del profesorado recogeremos las notas, asistencia y si el alumno hace entrega o no de los trabajos que se mandan. También tendrá como opcionales la atención, motivación y la nota del alumno en los entregables. Luego, el alumno tendrá como tarea opcional hacer un feedback rápido en el cual podrá ofrecer a la aplicación su atención en clase, motivación, asistencia y tiempo dedicado a las asignaturas fuera del horario de clase. Es a tener en cuenta que si los datos opcionales se entregan, se conseguirá un análisis mucho más preciso y eficaz, por eso son importantes unas cuantas claves que más tarde se mencionan en el apartado de actividades y recursos clave.\\
\\
\textbf{Resultados - Alarmas:}
\paragraph{}
Teacheck tendrá un sistema de alarmas que funcionará según el resultado de los análisis. Estas alarmas servirán para avisar a los profesores, vía email y aplicación, sobre un estado crítico o sobre la necesidad de darle un toque a un alumno. Entonces, si después de un análisis la aplicación detecta una de las distintas alarmas que tenemos diseñadas, las cuales se dividen en dos grupos, avisará al profesor o responsable correspondiente. Tal y como se acaba de mencionar las alarmas están divididas en dos grandes grupos, uno para alertar de un alumno y otro para alertar de un posible problema a nivel clase. Por parte de las alertas del alumno, se recibirán cuando se note un bajo rendimiento tanto en una asignatura como en el curso en general. Luego se analizará el rendimiento general de una clase por lo tanto, en caso de que baje demasiado saltará una alerta, a la vez que con el nivel de satisfacción, que en caso de que el nivel sea bajo también se podrá detectar. Por último la aplicación tendrá un calendario por curso donde los profesores introducirán sus exámenes correspondientes para que así la aplicación pueda realizar un escaneo una semana antes del mismo y así el profesor pueda ver si el alumno está preparado o no.\\
\\
\textbf{Seguridad:}
\paragraph{}
El producto desarrollado tiene que cumplir unos requisitos de seguridad y por ello se realizará tanto un análisis de riesgos como un análisis de riesgos residual, para luego implementar lo definido y contar con una aplicación seguro y libre en lo medida de lo posible de amenazas.\\
\\
\textbf{Mercado:}
\paragraph{} Teacheck se situa en la categoría de aplicaciones que intentan prevenir el deterioro del rendimiento académico de los alumnos. Dentro de esta categoría ya existe una institución que realiza este tipo de actividad. Pero Teacheck ofrece algo más que solamente el análisis de los datos proporcionados por el profesor, también ofrece la posibilidad de que un alumno pueda proporcionar datos los cuales permitirá a la aplicación ser más precisa y eficaz.
Ya que de esta manera, los datos introducidos por los profesores nos
dirán en qué está fallando el alumno y los datos proporcionados por el
alumno, cómo solucionarlo.

\section{Propuesta de Valor}

\paragraph{}
Teacheck es una aplicación con el objetivo de facilitar un seguimiento
en beneficio tanto del alumnado principalmente como del profesor. Con
esto pretendemos resolver el problema que actualmente hay con el alto
porcentaje de repetidores y suspensos. Creemos que la mayoría de estos
casos suceden por falta de responsabilidad y desconocimiento de cuando
el alumno va mal o su rendimiento no es el adecuado, y gran parte de
estos casos son evitables. Y para dar solución a esto el primer paso
es el interés del profesorado en intentar revertir esta situación y
tratar de alertar al alumno de su situación, que aunque él ya sea
consciente de ello en la mayoría de los casos, el hecho de recibir un
aviso o consejo de forma adecuada, esto es, teniendo en cuenta cual es
la manera correcta de decir y plantear los problemas, puede hacerle
cambiar. Por lo tanto, para conseguir esto es necesario el seguimiento
del alumno en cuestión y poder tanto ver cómo tener presente sus
notas, asistencia, motivación y sus entregables entre otros. Pero aquí
se nos plantea otro problema, y es que hacer el seguimiento de un
alumno es fácil, pero no es lo mismo con diez alumnos, o veinte, o
treinta. Entonces se complican las cosas ya que el profesor o tutor en
cuestión no podrá cumplir con el seguimiento de todos y los avisos no
podrán llegar a tiempo.

Pero para todo existe una solución y en este caso ofrecemos Teacheck,
tal y como hemos comentado al principio, una aplicación para realizar
el seguimiento del alumnado, de forma automática y precisa. El
objetivo es reducir en gran parte los repetidores y suspensos para así
aumentar el rendimiento de los alumnos. También vamos a poder analizar
diferentes problemas que puedan surgir a nivel de clase gracias a los
diferentes datos que almacenaremos. Con todo esto conseguiremos
aumentar el rendimiento de las clases y en consecuencia el de la
universidad en general, afectando positivamente tanto en su prestigio
como en su valor como institución lo que atraerá a nuevos alumnos y
empresas.

\section{Actividades Clave}

\paragraph{}
Tras un análisis detallado de la institución en la que se va a
desarrollar el sistema creemos que los puntos clave a la hora de
implementarlo y que aportarán valor, son los siguientes:

\begin{itemize}
\item \textbf{Atributos a analizar:} Se debe definir y concretar los
  atributos que se tendrán en cuenta en el machine learning, ya que
  estos serán los que en un futuro se valorarán y relacionarán entre
  ellos para sacar conclusiones tanto de las clases como de los
  alumnos.

\item \textbf{Alarmas:} Las alarmas que los atributos mencionados en
  el punto anterior podrán llegar a generar deben ser claras y
  concisas, detectando así la raíz del problema a tiempo y aportando
  un punto de inicio a la hora de solventar el problema.

\item \textbf{Informar y comunicar:} Creemos que lo primero es
  informar correctamente al alumno de lo que esta aplicación es y lo
  que le puede aportar. No es algo creado para controlarlo, si no algo
  que lo beneficiará si lo usa. Apenas le pide tiempo, solo unos pocos
  minutos a la semana y es importante que el alumno entienda esto para
  evitar malentendidos y descontentos. La aplicación seguirá
  funcionando sin sus aportaciones pero son esenciales para que este
  funcione al 100\%. Además de esto, también es importante una vez
  salte un aviso, comunicarle lo ocurrido al alumno de forma correcta,
  ya que si no, no conseguiremos revertir la situación tal y como
  queremos que suceda.

\nocite{marcoPedagogico}\nocite{hezkuntzaEreduArdatzak}\nocite{metodologiaParticipativa}\nocite{aprendizajeBasadoEnProblemas}\nocite{profesorUniversitario}\nocite{modeloUniversitario}

\end{itemize}
