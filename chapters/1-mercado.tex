\chapter{\mercado}

\paragraph{}
La aplicación está dirigida a universidades que busquen hacer un seguimiento detallado de su alumnado. El usuario final del sistema, por tanto, serán tanto los diferentes grupos de profesores o coordinadores como los alumnos que se encuentren en la institución. El ámbito geográfico que pretendemos abarcar es el de nivel nacional. Tras analizar diferentes aplicaciones hemos visto que se dividen en distintas categorías:

\begin{itemize}
\item Aplicaciones con IA que ayudan a los alumnos a aprender de una manera más eficiente y efectiva.
\begin{itemize}
\item \textbf{Easy learning:} Kidaptive es una plataforma de enseñanza adaptativa que impulsa una variedad de dominios de aprendizaje que incluyen dos aplicaciones creadas por Kidaptive para el aprendizaje temprano. Osmo es un juego interactivo que combina aprendizaje online y experiencial.
\end{itemize}
\item Aplicaciones que ayudan a los profesores en la enseñanza con la ayuda de la IA.
\begin{itemize}
\item \textbf{Contenido:} Los proveedores de contenido premium utilizan cada vez más el aprendizaje automático para ofrecer la siguiente mejor lección. Por ejemplo, startups como Content Technologies Inc. hacen uso de machine learning para automatizar su producción y automatización de procesos de negocio, diseño instruccional y soluciones de contenido y el proceso de enseñanza. 
\end{itemize}
\item Aplicaciones sin IA que automatizan las actividades de monitoreo del profesor en respecto al alumnado.\cite{appsEvaluacionEstudiantes}
\begin{itemize}
\item \textbf{Additio:} Se trata de una herramienta versátil con muchas funcionalidades al alrededor del mundo educativo, entre ellas la capacidad de llevar un registro de notas de los estudiantes de forma muy visual, intuitiva y práctica.
\item \textbf{TeacherKit:} Permite crear diferentes clases, cada una con sus alumnos y un sinfín de opciones para cada una de ellas. TeacherKit ayuda a llevar un registro de notas y también de asistencias y de comportamiento, con la posibilidad de exportar todos los datos para gestionarlos por su cuenta.
\end{itemize}
\item Aplicaciones con IA que monitorean el rendimiento de los alumnos y sacan alarmas según los diferentes objetivos:
\begin{itemize}
\item Aplicaciones que tienen como objetivo prevenir el abandono de alumnos en respecto a su carrera.
\begin{itemize}
\item \textbf{Universidad de Derby:} donde se implementó un sistema de monitoreo de la deserción estudiantil que utiliza los datos para predecir qué estudiantes tienen riesgo de dejar sus estudios, permitiéndole a la institución intervenir antes de que ello suceda.\cite{riesgoDejarEstudios}
\end{itemize}
\item Aplicaciones que previenen el deterioro del rendimiento del alumnado con el fin de revertir la mala situación para obtener mejores resultados.
\begin{itemize}
\item \textbf{Universidad Internacional de la Rioja:} Un equipo de expertos de la Universidad Internacional de La Rioja trabaja en un proyecto piloto para, gracias al uso y aplicación de la Inteligencia Artificial (IA), poder medir ,mediante algoritmos, dicho rendimiento. Se analiza el comportamiento del alumno en la plataforma, su participación en los foros, su interacción con el material de estudio, las calificaciones intermedias obtenidas en la evaluación continua… Al poder compararse con los históricos de estudiantes anteriores, se observa si existe un patrón.\cite{rendimientoAlumnos}
\end{itemize}
\end{itemize}
\end{itemize}

\paragraph{} Teacheck se situa en la categoría de aplicaciones que intentan prevenir el deterioro del rendimiento académico de los alumnos. Dentro de esta categoría ya existe una institución que realiza este tipo de actividad. Pero Teacheck ofrece algo más que solamente el análisis de los datos proporcionados por el profesor, también ofrece la posibilidad de que un alumno pueda proporcionar datos los cuales permitirá a la aplicación ser más precisa y eficaz.
Ya que de esta manera, los datos introducidos por los profesores nos dirán en qué está fallando el alumno y los datos proporcionados por el alumno, cómo solucionarlo.




































% line in order to check if utf-8 is properly configured: áéíóúñ
