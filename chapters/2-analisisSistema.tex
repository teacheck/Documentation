\chapter{Análisis del Sistema}
% \thispagestyle{empty}
\section{Arquitectura del sistema}
\irudia{template/figs/sys-arch.jpg}{1}{Arquitectura del Sistema}
\paragraph{}
\begin{itemize}
  \item{\textbf{Web Application:}} La aplicación web tiene el objetivo
    de visualizar los resultados proporcionados por el sistema de
    análisis de datos y monitorear los diferentes alumnos registrados
    en el sistema de Teacheck.\nameref{webAplication}
  \item{\textbf{Keycloak:}} Keycloak es un servicio que gestiona el
    acceso y identidad de entidades. Este servicio acoplado con la
    aplicación web para proveer autenticación y autorización a
    recursos.
  \item{\textbf{Servicio Machine Learning:}} El servicio de análisis
    de datos y predicción. Este sistema tiene el objetivo de analizar
    todos los datos proporcionados por la universidad cliente en
    respecto a los alumnos. Contiene un núcleo que se trata de un
    servicio más pequeño que hace las predicciones semanalmente.\nameref{machineLearning}
  \item{\textbf{Servicio Conversor:}} Servicio de conversión de
    formatos. El sistema no obliga al cliente adaptarse al formato de
    intercambio de información que dispone Teacheck. Este servicio
    trata de convertir todos los datos proporcionados por el sistema
    de la universidad cliente al formato adecuado que necesita nuestro
    sistema.\nameref{conversor}
  \item{\textbf{Servicio Mensajería:}} El servicio de mensajería trata
    transmitir mensajes desde un dispositivo que controla asistencia
    de alumnos a nuestro sistema. La implementación hace el equipo de
    Teacheck. Este servicio es adicional y no es obligatorio para el
    funcionamiento del sistema.\nameref{servicioMensajería}
  \item{\textbf{Servicio Base de Datos:}} Este es el servicio central
    que se comunica con la base de datos de Teacheck. Se encarga de
    todas la transacciones y consultas del sistema.\nameref{accesoADatos}
\end{itemize}
\section{Diseño centrado en el usuario}
Una buena aplicación necesita un buen diseño, pero para que sea bueno
no es suficiente hacerlo bonito, también es necesario hacerlo
útil. Para asegurarnos de esto, hemos seguido lo que se llama Diseño
centrado en el usuario. DCU es una metodología enfocada a las
necesidades reales del usuario.
\subsection{Contexto de uso}
Primero debemos analizar los diferentes tipos de usuario que la
aplicación tiene y como interactúan ellos con la aplicación.

Algunas partes de la aplicación pueden, y probablemente serán
complejas, pero eso no debería ser un problema. La complejidad no
siempre significa dificultad, a veces las tareas complejas se pueden
realizar fácilmente. Tiene que ser una herramienta poderosa, que nos
aporte mucho, pero a la vez, que sea fácil de utilizar.
\subparagraph{Entorno}
Esta aplicación está diseñada para Universidades y son los los
integrantes de esa universidad los que la utilizarán. Debe ser una
herramienta útil que nos proporcione la información necesaria para
conseguir los objetivos. Por lo tanto, la eficiencia es mucho más
importante que la estética.
\subsection{Usuarios}
\begin{itemize}
\item \textbf{Alumnos:} Es una aplicación en la que los alumnos no
  pasarán mucho tiempo ya que no tendrá muchas funcionalidades. Por
  una lado, los alumnos podrán ver su perfil (foto, datos personales,
  estadísticas generales,…). Por otro lado, podrán ver el estado en el
  que están, es decir, estadísticas e información más ecisa tales como
  estado del curso, sus notas, asistencia, entregables, etc.  Y por
  último, podrán realizar las encuestas semanales. Estas encuestas son
  rápidas y sencillas pero que serán muy útiles a la hora de analizar
  a los alumnos.

Cada año, nuevos alumnos comenzarán a utilizar la aplicación y otros
que ya la utilizaban dejarán de hacerlo al terminar sus estudios. Por
lo tanto, tiene que ser una aplicación fácil de aprender y de
utilizar.
\item \textbf{Profesores:} Probablemente, son los usuarios que más
  utilizarán la aplicación. Estos, tendrán disponibles tres
  apartados. En el primero, podrán ver el estado de todos los alumnos
  de cada curso donde da clase. En el segundo, tendrá el apartado de
  alarmas donde verá que alumnos suyos están en peligro o necesitan un
  toque de atención. Y por último, tendrá el apartado de las
  estadísticas generales, es decir, se le mostraran los resultados
  generales de las encuestas a nivel de clase.
\item \textbf{Coordinadores del curso:} Los coordinadores del curso,
  al igual que los profesores, también pasarán tiempo con esta
  aplicación. Y al igual que los profesores, podrán ver el estado de
  los alumnos de el curso o los cursos que coordina, el apartado de
  las alarmas y los resultados generales de las encuestas
  semanales. Pero a diferencia de los profesores, las alarmas que se
  le notifiquen no serán solo a nivel de alumno. También se le
  avisarán de los alarmas a nivel de clase.
\end{itemize}
\subsection{Requisitos}
\subsubsection{Requisitos de negocio}
\paragraph{\textsc {General}}
\begin{itemize}
\item {La aplicación permitirá hacer diferentes cosas al usuario según
  el rol asignado.}
\item {La aplicación debe estar protegida por un login.}
\item {La aplicación guardará los datos de manera protegida ya que
  maneja datos delicados.}
\end{itemize}
\paragraph{\textsc {Alumnos}}
\begin{itemize}
\item {La aplicación tiene que ser capaz de visualizar estadísticas de
  los alumnos tales como el estado, notas, asistencia y entregables.}
\item {La aplicación debe ser capaz de mostrar el perfil al usuario
  con su foto y datos personales.}
\item {La aplicación permitirá rellenar a los alumnos encuestas
  semanales.}
\item {La aplicación hará análisis periódicos de los alumnos con las
  fechas definidas anteriormente.}
\end{itemize}
\paragraph{\textsc {Profesores}}
\begin{itemize}
\item {La aplicación permitirá a los profesores ver el el estado de
  los alumnos de las clases donde imparte clases.}
\item {La aplicación permitirá a los profesores ver el apartado de
  alarmas donde verá todos los alumnos a los que le ha saltado la
  alarma.}
\item {La aplicación mostrará a los profesores los resultados
  generales de las encuestas por cada asignatura que enseña.}
\item {La aplicación deberá enviar dichas alarmas a los profesores vía
  email.}
\end{itemize}
\paragraph{\textsc {Coordinadores del curso}}
\begin{itemize}
\item {La aplicación permitirá a los coordinadores ver el el estado de
  los alumnos del curso que coordina.}
\item {La aplicación permitirá a los coordinadores ver el apartado de
  alarmas donde verá todos los alumnos a los que le ha saltado la
  alarma y todas las alarmas a nivel de clase.}
\item {La aplicación mostrará a los coordinadores los resultados
  generales de las encuestas por cada curso que coordina.}
\item {La aplicación deberá enviar dichas alarmas a los coordinadores
  vía email.}
\end{itemize}
\subsubsection{Requisitos de usuario}
\paragraph{\textsc {Alumnos}}
\begin{itemize}
\item {Los alumnos tienen que ser capaces de ver su estado del curso
  mediante tablas, listas o dasboards.}
\item {Los alumnos tienen que ser capaces de acceder a su perfil
  mediante un simple botón situado a la izquierda. }
\item {Los alumnos tienen que ser capaces de rellenar encuestas
  semanales mediante formularios.}
\end{itemize}
\paragraph{\textsc {Profesores}}
\begin{itemize}
\item {Los profesores tienen que ser capaces de ver el estado de sus
  alumnos a través de listas, tablas de datos o dashboards.}
\item {Los profesores tienen que ser capaces de ver las alarmas de
  dichos alumnos mediante listas.}
\item {Los profesores tienen que ser capaces de ver los resultados de
  las encuestas mediante tablas o gráficos.}
\item {Los profesores tienen que ser capaces de acceder a su perfil
  mediante un simple botón situado a la izquierda.}
\end{itemize}
\paragraph{\textsc {Coordinadores del curso}}
\begin{itemize}
\item {Los coordinadores tienen que ser capaces de ver el estado de
  sus alumnos a través de listas, tablas de datos o dashboards.}
\item {Los coordinadores tienen que ser capaces de ver las alarmas de
  dichos alumnos mediante listas.}
\item {Los coordinadores tienen que ser capaces de ver los resultados
  de las encuestas mediante tablas o gráficos.}
\item {Los coordinadores tienen que ser capaces de acceder a su perfil
  mediante un simple botón situado a la izquierda.}
\end{itemize}
\subsubsection{Requisitos funcionales}
\begin{itemize}
\item {API de Highcharts. Esto se necesitará para crear gráficos con
  altos detalles de visualización de datos.}
\item {API’s de Vertx. Se utilizarán para la construcción del
  sistema. }
\item {La aplicación debe desarrollarse en el lenguaje de programación
  Java.}
\item {Se va a utilizar Docker para crear una imagen de la aplicación
  para implementarla en un servidor remoto.}
\item {La aplicación tiene que estar desarrollada en una plataforma
  multilenguaje.}
\end{itemize}
\subsubsection{Requisitos de calidad de servicio}
\begin{itemize}
\item {La aplicación estará disponible 24/7.}
\item {La aplicación debe ser escalable, es decir, al introducir
  nuevas asignaturas no debe influir en el modelo de la arquitectura.}
\item {El sistema debe ser tolerante a fallos, incluso si ocurre un
  problema de hardware, el sistema debe seguir funcionando como se
  espera.}
\item {La seguridad de la aplicación debe incluir autorización,
  autenticación, seguridad de acceso a datos y acceso seguro para el
  sistema de implementación, así como su administración.}
\item {El sistema debe ser mantenible, monitoreado y actualizado
  cuando sea necesario.}
\end{itemize}
\section{Casos de uso}
\irudia{template/figs/Alumno.jpg}{0.6}{Casos de uso - Alumno}
\irudia{template/figs/Profesor-Coordinador.jpg}{1}{Casos de uso -
  Profesor y coordinador}
