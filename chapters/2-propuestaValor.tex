\chapter{\propuestaValor}

\paragraph{}
Teacheck es una aplicación con el objetivo de facilitar un seguimiento en beneficio tanto del alumnado principalmente como del profesor. Con esto pretendemos resolver el problema que actualmente hay con el alto porcentaje de repetidores y suspensos. Creemos que la mayoría de estos casos suceden por falta de responsabilidad y desconocimiento de cuando el alumno va mal o su rendimiento no es el adecuado, y gran parte de estos casos son evitables. Y para dar solución a esto el primer paso es el interés del profesorado en intentar revertir esta situación y tratar de alertar al alumno de su situación, que aunque él ya sea consciente de ello en la mayoría de los casos, el hecho de recibir un aviso o consejo de forma adecuada, esto es, teniendo en cuenta cual es la manera correcta de decir y plantear los problemas, puede hacerle cambiar. Por lo tanto, para conseguir esto es necesario el seguimiento del alumno en cuestión y poder tanto ver cómo tener presente sus notas, asistencia, motivación y sus entregables entre otros. Pero aquí se nos plantea otro problema, y es que hacer el seguimiento de un alumno es fácil, pero no es lo mismo con diez alumnos, o veinte, o treinta. Entonces se complican las cosas ya que el profesor o tutor en cuestión no podrá cumplir con el seguimiento de todos y los avisos no podrán llegar a tiempo. 

Pero para todo existe una solución y en este caso ofrecemos Teacheck, tal y como hemos comentado al principio, una aplicación para realizar el seguimiento del alumnado, de forma automática y precisa. El objetivo es reducir en gran parte los repetidores y suspensos para así aumentar el rendimiento de los alumnos. También vamos a poder analizar diferentes problemas que puedan surgir a nivel de clase gracias a los diferentes datos que almacenaremos. Con todo esto conseguiremos aumentar el rendimiento de las clases y en consecuencia el de la universidad en general, afectando positivamente tanto en su prestigio como en su valor como institución lo que atraerá a nuevos alumnos y empresas.































% line in order to check if utf-8 is properly configured: áéíóúñ
