\chapter{\actividadesClave}

\paragraph{}
Tras un análisis detallado de la institución en la que se va a desarrollar el sistema creemos que los puntos clave a la hora de implementarlo y que aportarán valor, son los siguientes:  
 
\begin{itemize}
\item \textbf{Atributos a analizar:} Se debe definir y concretar los atributos que se tendrán en cuenta en el machine learning, ya que estos serán los que en un futuro se valorarán y relacionarán entre ellos para sacar conclusiones tanto de las clases como de los alumnos.

\item \textbf{Alarmas:} Las alarmas que los atributos mencionados en el punto anterior podrán llegar a generar deben ser claras y concisas, detectando así la raíz del problema a tiempo y aportando un punto de inicio a la hora de solventar el problema.

\item \textbf{Informar y comunicar:} Creemos que lo primero es informar correctamente al alumno de lo que esta aplicación es y lo que le puede aportar. No es algo creado para controlarlo, si no algo que lo beneficiará si lo usa. Apenas le pide tiempo, solo unos pocos minutos a la semana y es importante que el alumno entienda esto para evitar malentendidos y descontentos. La aplicación seguirá funcionando sin sus aportaciones pero son esenciales para que este funcione al 100\%. Además de esto, también es importante una vez salte un aviso, comunicarle lo ocurrido al alumno de forma correcta, ya que si no, no conseguiremos revertir la situación tal y como queremos que suceda.

\nocite{marcoPedagogico}\nocite{hezkuntzaEreduArdatzak}\nocite{metodologiaParticipativa}\nocite{aprendizajeBasadoEnProblemas}\nocite{profesorUniversitario}\nocite{modeloUniversitario}
  
\end{itemize}






























% line in order to check if utf-8 is properly configured: áéíóúñ
