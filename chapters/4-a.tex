\chapter{Lo de IA}

\section{Introducción}
\paragraph{}
El núcleo de la aplicación reside en una máquina la cual tiene tres diferentes objetivos. El primero, predecir el rendimiento de un alumno a lo largo del curso, creando alarmas cuando sean necesarias. El segundo, poder analizar y descubrir los detonantes de las posibles alarmas, ya que sirve de poco saber de la existencia de un problema sin conocer las razones. Y por último, poder a su vez predecir cuando el rendimiento de una clase a decaído, y al igual que con un alumno, poder crear una alarma de dicho problema.
\paragraph{}
Dichos objetivos se han logrado haciendo uso del aprendizaje automático, una rama de la inteligencia artificial cuyo objetivo es desarrollar técnicas que permitan que las computadoras aprendan. De esta manera, hemos generado modelos, extraído correlaciones y desentrañado clasificadores de datos como el bayesiano ingenuo(1). Pero antes de entrar en detalle y ver cómo se ha analizado, diseñado e implementado todo lo mencionado, vamos a dar un paso atrás, ya que para poder desarrollar todo lo mencionado, se necesitan datos.
\paragraph{}
Dichos datos son necesarios tanto para crear los modelos que se usarán como predictores, como a su vez para luego testearlos y hacer funcionar nuestra aplicación. Además, no son datos generados aleatoriamente, si no que tienen que seguir una lógica, una lógica que heredará el predictor inteligente, y a su vez toda la aplicación. Por lo tanto, se ha realizado un análisis de los datos que se han querido generar, y del método utilizado para su generación.

\section{Datos}
\subsection{Variables}
\paragraph{}
Tal y como se acaba de mencionar, los datos cuentan con una gran relevancia, y debido a eso, cada dato o atributo se analiza, define y genera de forma individual. Dichos datos o atributos pertenecen a cada asignatura de un alumno, esto es, contamos con nueve atributos por asignatura, siendo cinco las asignaturas por semana y diez las semanas por curso. Los atributos son los siguientes:

\begin{itemize}
\item \textbf{Notas de competencias:} 
El primer atributo para analizar serán las notas conseguidas por los alumnos en los diferentes puntos de control de cada competencia. Este atributo será un decimal que nos ayudará a saber el desempeño del alumno en cuanto a un temario y tendrá un rango del 0 al 10.
\item \textbf{Asistencia al curso:}
\paragraph{}
 El segundo atributo que se tendrá en cuenta a la hora de realizar el análisis será la asistencia de un alumno a una asignatura a lo largo de la semana. Esta característica creemos que es importante ya que, aunque para cada universidad influya de una forma diferente en cuanto al desempeño del alumnado, es un atributo directamente relacionado con el entendimiento de las clases impartidas. Este dato será un porcentaje en relación con las horas de clase impartidas en un semana de una asignatura y la asistencia de un alumno a las mismas.
\item \textbf{Motivación del alumno:}
\paragraph{}
Este tercer atributo se ha decidido que sea opcional, esto es, en caso de que el profesor no vea nada excepcional en un alumno, no tendrá la necesidad de aportar ninguna información sobre el o ella. Está pensado para aquellas situaciones en las que el personal docente observe algún tipo de variación en la motivación del alumno. Por ello, será un número con un rango del 1 al 5 en el que un 1 sería una motivación nula en clase y 5 un comportamiento ejemplar del alumno.
\item \textbf{Atención del alumno:}
\paragraph{}
Además, en cuanto a la información recogida por el profesor, el cuarto atributo que tendremos en cuenta será la atención de un alumno en una asignatura a lo largo de la semana. Se ha comprobado que, aunque el atributo de asistencia esté cerca del 100\% si un alumno no presta atención en clase, las probabilidades de que ese sujeto no esté entendiendo el contenido de la asignatura son peligrosamente altas. El formato del dato será idéntico a la motivación.
\item \textbf{Entregables o trabajos de la asignatura:}
\paragraph{}
Este atributo estará relacionado con los diferentes trabajos a entregar en una asignatura en concreto. Se decidió tener en cuenta esta característica ya que los entregables están relacionados con el contenido práctico de una asignatura, y, por tanto, en caso de haber realizado el trabajo, las probabilidades de que un alumno se desempeñe correctamente en esa asignatura son muy altas. Esta característica solo nos dirá si se ha realizado dicha práctica, por tanto, será un booleano.
\begin{itemize}
\item{Nota de los entregables o trabajos de la asignatura:} Este atributo únicamente será rellenado en caso de que el atributo anterior sea verdadero y será un número con un rango del 0 al 10.
\end{itemize}
\item \textbf{Motivación:}
\paragraph{}
Al igual que el personal docente puede saber la motivación de un alumno en clase, el propio alumno también será capaz de realizar una encuesta y decirnos si realmente está motivado en una asignatura o no. Este dato será ingresado desde la encuesta semanal que el alumnado tendrá la oportunidad de realizar al finalizar la semana. Los datos ingresados serán idénticos a la motivación ingresada por el personal docente. Está comprobado que dado el anonimato desde el que se ingresará esta información, el alumnado podrá sincerarse y hacer una valiosa aportación al sistema.
\item \textbf{Atención:}
\paragraph{}
Como el atributo anterior, en este caso, el alumnado será capaz de ingresar este tipo de información desde la encuesta al finalizar la semana.  Nos interesa tanto la atención que un alumno pueda tener en clase desde el punto de vista del profesor y del alumno, ya que, en algunos casos, pueda parecer que un alumno no esté prestando atención y realmente si lo esté haciendo y viceversa.
\item \textbf{Horas aplicadas a la asignatura fuera de clase:}
\paragraph{}
Por último, sabemos que la universidad requiere de una cantidad de horas de estudio fuera de la misma. Es un número subjetivo, ya que cada alumno necesitará de diferente tiempo para entender una asignatura. Pero gracias a las medias que podamos conseguir de una clase entera, sabremos si un alumno holgazanea al llegar a casa o realmente repasa el contenido dado en clase.
\end{itemize}

\subsection{Horizonte de análisis}
\paragraph{}
Haciendo uso de estos datos, tal y como se ha mencionado anteriormente, se predice el rendimiento de cada alumno en cada una de las asignaturas en los que se encuentra matriculado durante el curso. Dicho curso cuenta con diez semanas, y cada semana con un modelo diferente. Esta división se debe a que el número de datos en las primeras semanas y en las últimas varía considerablemente. Por lo tanto, no todos los modelos difieren entre sí, pero sí que hay una evolución en los modelos a medida que avanza el curso.
\paragraph{}
Los datos o conclusiones que sacaremos serán los siguientes. Primero, en cuanto al rendimiento del alumno, se determinará por asignatura si el alumno requiere de una alarma o no. Luego, se detalla el impacto de cada atributo a la hora de decidir la alarma, lo cual se utiliza para sacar el detonante de la misma. Por último, usamos las correlaciones y el número de alarmas para determinar y generar las alarmas de clase. Más adelante se detalla el procedimiento de cada una de ellas.


\subsection{Creación datos}
\paragraph{}
Una vez definidas las variables hemos creado los datos que se usarán para diseñar y crear los modelos. La creación de dichos datos no ha sido aleatoria, ya que aun siendo una simulación se requerían datos que cumpliesen una lógica para de esta manera poder crear modelos que cumpliesen con su función.

Dicho esto, los datos se han generado utilizando Excel(2), básicamente debido a su facilidad de generar muchos datos haciendo uso de sus macros. Por ello, se ha utilizado una macro para cada variable, y estas se han generado siguiendo una distribución normal que contaba con una desviación típica. De esta manera, generamos cientos de datos aleatorios que seguían una lógica definida por nosotros, la cual definimos tras estudiar y comparar datos reales.

Hablando de números, se han creado datos de 300 alumnos para diez semanas de lo que podría ser un curso, contando cada semana con cinco asignaturas. En las dos primeras semanas tanto la nota del examen como el entregable están vacías, y de la tercera a la quinta se añaden datos tanto al entregable como a su nota. Esto se hizo con el objetivo de tener datos más reales, ya que en las primeras semanas de un curso real no se suelen contar con estos datos.

Una vez generados todos los datos para las diez semanas, comenzamos a diseñar lo que sería la función que más tarde generaría el target, esto es, lo que definiría si el alumno está necesitado de una alarma o no. Para ello, dimos un peso a cada variable, creamos la función, y después de insertar la misma en una macro de Excel, generamos el target.

Y de esta manera, haciendo uso de una distribución normal seguida de una desviación típica y añadiendo la función diseñada por nosotros, generamos los datos de forma lógica dándoles el sentido que buscábamos y necesitábamos para poder crear nuestros modelos.

(aquí se puede insertar foto o tabla o lo que sea con las distribuciones que hemos seguido. Tmb para la función.)


\subsection{Creación de modelos}
\paragraph{}
Tras generar los datos, continuamos con la creación de los modelos anteriormente mencionados, esto es, uno por semana.

Los dos primeros modelos no cuentan ni con la nota del examen ni con el entregable, y los siguientes tres siguen sin contar con la nota del examen. Una vez llegados a la quinta semana, se cuenta con todos los datos. Es cierto que no son diez modelos únicos, pero la decisión de crear todos ellos se basa en el hecho de que en un caso real se requeriría un modelo personalizado e individual por semana.

Para generar los modelos se utilizó el clasificador bayesiano ingenuo, debido a que de las pocas opciones disponibles, siendo la diferencia a la hora de predecir tan insignificante, era con la que más familiarizada estábamos. Para la creación de dichos modelos, utilizamos el 80\% del dataset antes mencionado, esto es, 240 de los 300 alumnos. El 20\% restante se usó para el testeo de los modelos, y más tarde para insertarlos en la base de datos a modo de nuevos alumnos. Los resultados fueron positivos, el porcentaje de acierto rondaba el 85-90\%, lo cual cumplía con las expectativas. Además de ello, se consiguen las correlaciones para después poder realizar un análisis y determinar si se requiere de una alarma de clase o no. Si el número de alarmas en una clase supera el 50\%, saltará una alarma a nivel de clase y los detonantes serán las correlaciones antes mencionadas.


\section{Enseñanza automática}
\subsection{Predicciones y correlaciones}
\paragraph{}
Con el objetivo de lograr resultados haciendo uso de los modelos previamente mencionados, se decidió crear un servicio web en R, el cual una vez está activo y escuchando, recibe información en formato JSON relativa a 60 estudiantes por cada una de las 5 asignaturas en una semana.  

Se requiere que unos de los datos que va a ser enviado a este servicio junto con todos los atributos del alumno, sea el número de la semana, para así poder saber cuál es el modelo a aplicar en cada semana.

Por lo tanto, el procedimiento una semana cualquiera, será el siguiente:
Para empezar, la información previamente definida del alumnado debe ser enviada al servicio web en marcha. Luego, esta información es formateada dependiendo del tipo de dato que tenga cada atributo, ya que para poder realizar predicciones y correlaciones todos los atributos deben ser numéricos y los valores nulos necesitan ser traducidos a un formato para que, a la hora de predecir, se tengan en cuenta como el valor que son, esto es, un 0. Una vez la información pueda ser interpretada por el predictor, se utiliza el modelo generado para la semana de la que provenga la información, y se predice la clase, nuestro caso, es un atributo con dos posibles valores “Alarma” y “No Alarma”.

Una vez tenemos una columna con esta información, es el momento de sacar, a nivel general, esto es de los 60 alumnos, el peso que ha tenido cada atributo a la hora de tomar la decisión. Para ello simplemente son comparados cada atributo de todos los alumnos con la nueva columna de predicciones que acabamos de generar. Estos llamados coeficientes de correlaciones, nos indicarán globalmente qué atributo ha sido decisivo a la hora de indicar la alarma y, por tanto, todos y cada uno de los atributos del alumnado tendrán un porcentaje con su nivel de influencia. Estos coeficientes de correlaciones pueden ser calculados con diferentes métodos: el coeficiente de correlación de concordancia Kendall, el coeficiente de correlación de Pearson y el coeficiente de correlación de Spearman. En nuestro caso, calcularemos el coeficiente de correlación de Kendall (8) ya que contamos con ambos tipos de valores, ordinales e intervalos.
Cada semana, tanto las predicciones como las correlaciones variarán, ya que teniendo en cuenta el modelo se generarán X alarmas, y en función de estas alarmas, las relativas correlaciones.
	
Desde una perspectiva más general, por tanto, a este script  se le envía un mensaje en formato JSON relativo a 60 alumnos con toda la información recopilada en una semana por cada asignatura, exceptuando la clase, y nos devolverá un resultado con la clase y el id de cada alumno. Además, en otra lista, los 9 atributos del alumnado cada uno con el valor de su correlación respecto a la clase.

\subsection{Clasificador bayesiano ingenuo}
\paragraph{}
Una vez conseguidos los resultados anteriores, se estudió la posibilidad de encontrar el motivo por el cual a cada alumno se le había sido asignado un valor de la clase u otro, esto es, el motivo por el cual una alarma había saltado o no. Gracias a las correlaciones pudimos saber qué tendencia seguía una clase por asignatura, pero no cómo el modelo había llegado a asignar cada valor a la clase.

Por lo tanto, después de analizar y definir cómo funcionaba el método del bayesiano ingenuo que usamos para generar los modelos y más tarde realizar las predicciones, se empezó a aplicarlo al dataset creado para la generación de los modelos y así poder conocer el impacto individual de cada atributo a la hora de decidir la clase.

Antes de explicar cómo se aplicó dicho método bayesiano al dataset utilizado en este proyecto, se realizará un pequeño resumen del funcionamiento del clasificador bayesiano ingenuo. Este clasificador, tiene en cuenta cada posible valor que puede tener un atributo, así pues, en nuestro caso, por ejemplo, la atención es un atributo cuyos valores posibles son un numérico del 1 al 5. Una vez tenemos estos posibles valores, es necesario conseguir la suma del total acumulado para cada valor, o, lo que es lo mismo, cuántas veces ha sido repetido un mismo valor entre el dataset para un atributo en concreto. Este total acumulado nos indica la presencia de ese valor dentro de un atributo. Además, este método, como su nombre indica, clasifica el dataset en tantos grupos como valores posibles tenga la clase. Este trabajo es necesario ya que más tarde el total acumulado conseguido anteriormente será dividido entre el total acumulado del valor de la clase al que pertenezca. En conclusión, esta división indicará para cada valor de un mismo atributo, que influencia o peso tiene sobre la clase a la hora de decidir su valor. Las predicciones que genera este clasificador utilizan este porcentaje de influencia de cada valor para cada atributo y el porcentaje de influencia de un valor de la clase entre el total de valores de la clase. En total, nos genera tantos valores como posibles valores tenga la clase, y el máximo entre ellos será la predicción final.

\irudia{template/figs/naiveBayes.jpg}{1}{Ejemplo de predicción NaiveBayes}

\paragraph{}
Así pues, estos fueron los pasos necesarios para poder aplicar este método a nuestro dataset, el utilizado para generar los modelos del clasificador:
\begin{enumerate}
\item Separar el dataset entre los alumnos con alarmas y los que no. Descartar los que no tenían alarma.
\item Calcular para cada valor de cada atributo, cuantas veces se repetía a lo largo de los 240 alumnos. Incluido el valor de la clase, “Alarma” respecto a su otro valor “No Alarma” esto es, el número de alarmas esa semana.
\item Una vez conseguimos esto, realizar la división entre el total acumulado para cada valor de cada atributo entre el total acumulado del valor “Alarma” de la clase.
\item El resultado será, para cada posible valor de un atributo, un peso o porcentaje a la hora de generar la alarma.
\item Repetir este proceso por 10 semanas.
\end{enumerate}

\paragraph{}
Cabe mencionar que en nuestro caso solamente nos interesaba saber el motivo por el que se había generado el valor “Alarma” de la clase, ya que las predicciones habían sido calculadas automáticamente mediante el script de R.

Nos interesaba mantener esta información ajena a cualquier otro dato de este proyecto y, por ello, se generó un fichero por semana al que se accederá mediante el servicio de enseñanza automática que se definirá más adelante.

Así pues, cada vez que un alumno tenga una alarma, se tendrá en cuenta en qué semana se encuentra y se accederá a dicho fichero para saber qué peso tiene cada valor de cada atributo de ese alumno a la hora de generar una alarma esa semana. Además dado que existen ciertos atributos cuyos valores oscilan entre el 0.00 y el 9.99 no se ha podido llegar a contemplar todos y cada unos de los valores, ya que en el dataset inicial solo se recogen muestras de 300 alumnos y entre ellos los valores que llegan a repetirse no cubren todo el abanico de posibles valores para ese atributo.

\irudia{template/figs/vecesRepetidas.jpg}{1}{Ejemplo de semana con las veces repetidas por cada valor de atributo.}
\irudia{template/figs/porcentajeValorAtributo.jpg}{1}{Ejemplo de semana con los porcentajes por cada valor de atributo (fichero de semana 5).}

\subsection{Conclusiones}
\paragraph{}

Evolución de las alarmas de un alumno:
Como hemos mencionado anteriormente el script en R nos indicará que alumnos tiene alarmas y cuáles no, por tanto, a lo largo del año podremos construir un gráfico viendo que semanas han sido más críticas para un alumno en concreto.
            Evolución de las alarmas de una clase:
Al igual que en el apartado anterior podremos apreciar qué asignaturas o qué semanas han sido las peores en cuanto al rendimiento de toda una clase.
            Evolución de la influencia de cada atributo a la hora de generar alarmas:
Con la intención de conocer en cada semana cuáles han sido los atributos que han acarreado alarmas en general y cuales han sido los más importantes. Haciendo uso de las correlaciones, podremos realizar gráficos de la evolución por atributo y/o semana.
            Evolución de los atributos que han generado una alarma en un alumno:
Ya que sabemos los valores de los atributos para cada alumno cada semana, podremos saber qué influencia han tenido estos a la hora de generar una alarma de forma personalizada.

\subsection{Análisis de los resultados}
\paragraph{}
\textbf{Ejemplo nuevo alumno en semana 3 con alarma:}
\irudia{template/figs/tablaEjemploAlumno.jpg}{1}{}

\paragraph{}
\paragraph{}
\textbf{Resultado del programa teniendo en cuenta el fichero de la semana 3:} 
Atributo:Asistencia Valor:66.68 Peso:0.02173913
Atributo:EntregableC11 Valor:VERDADERO Peso:0.630434783
NO EXISTENTE-Atributo:NotaEntregableC11 Valor:1.43 Peso:0.043478261
Atributo:HorasC11 Valor:1.42 Peso:0.043478261
Atributo:AtencionP1 Valor:2 Peso:0.239130435
Atributo:MotivacionP1 Valor:2 Peso:0.217391304
Atributo:AtencionA1 Valor:3 Peso:0.217391304
Atributo:MotivacionA1 Valor:3 Peso:0.152173913
Atributo:Criterios Valor:Alarma Peso:0.153846154

\paragraph{}
\textbf{Porcentaje de influencia respecto a la alarma ordenado por atributo:}
[EntregableC11=0.630434783, AtencionP1=0.239130435, AtencionA1=0.217391304, 
MotivacionP1=0.217391304, Criterios=0.153846154, MotivacionA1=0.152173913, 
NotaEntregableC11=0.043478261, HorasC11=0.043478261, Asistencia=0.02173913]

\paragraph{}
\textbf{Aplicamos parte de la fórmula del Clasificador Bayesiano Ingenuo:}

	0.63 * 0.23 * 0.21 * 0.21 * 0.15 * 0.04 * 0.04 * 0.02 * \textbf{0.15}

\paragraph{}
\textbf{Como podemos ver, los atributos que más han afectado a que a este alumno tenga un alarma son los siguientes:}
\begin{enumerate}
\item Entregable
\item Atención por parte del profesor
\item Atención por parte del alumno
\item Motivación por parte del alumno
\end{enumerate}

\paragraph{}
\textbf{Conclusión:}
Los valores de los atributos que son decimales no van a tener una gran influencia en la alarma ya que han sido analizamos únicamente 300 alumnos. Cuantos más grande fuera la muestra de alumnos, mayores las posibilidades de que estos valores se repitan y, por tanto, la influencia en la alarma sería más exacta. Si bien es cierto que este problema podría solucionarse mediante un sistema de valoración que se basa en nominales, como, “Alto”,”Medio”,”Bajo”...No sería realista ya que hoy en día las universidades valoran con una nota del 0 al 10, decimales incluidos.

\paragraph{}
\textbf{Variación de alarmas por semana:}
\paragraph{}
\irudia{template/figs/variacionAlarmas.png}{1}{Variación de alarmas por semana.}
Como podemos apreciar en este gráfico, el número de predicciones por semana aumenta drásticamente a medida que avanza el año. Además el salto más grande se puede observar en la sexta semana donde se han realizado ya los exámenes y por ende, los alumnos que hayan suspendido o hayan sacado una nota más baja de lo habitual, serán avisados a tiempo antes de que la situación empeore.

\paragraph{}
\textbf{Conclusión:}
Como  podemos apreciar tanto en la tabla como en los diferentes gráficos de las variaciones de las correlaciones por cada semana(9), algunos atributos como la motivación recibida por el alumno, indica que en las primeras semanas del curso, este atributo tiene una gran influencia para que el valor de la clase sea “Alarma” y que, por tanto, será un aspecto a tener en cuenta a la hora de dar clase ya que será necesario motivar a los alumnos las primeras semanas del curso ,pero no será tan importante al final.

En general, todos los gráficos indican que a partir de la semana 5, esto es, la mitad del curso transcurrido, la mayoría de los atributos tienen un gran influencia en la alarma. Sin embargo, hay un atributo en particular que tiene valores más altos que los demás, la nota(0.65). Y es que, al fin y al cabo, va a ser la que mayor peso tenga siempre a la hora de decidir si sacar una alarma o no por que será al final el atributo que indique si un alumno pasa o no de curso.Un estudio(9) indica que el coeficiente de correlación Kendall comparado a otros coeficientes como el Pearson o Spearman indica con un 0,7 lo que otros indican como un 0.9 de correlación por lo que la nota está cerca de decidir en algunas semanas directamente si un alumno tiene alarma o no.
La asistencia también es un atributo a tener en cuenta, pero que variará dependiendo de la centro en el que se implante el sistema. No todos los centro dan importancia a acudir a las clases lectivas.

Por último, destacar que los valores negativos en el coeficiente de correlación indican que a medida que ese atributo aumenta, el valor de la clase decrece, esto es, está más cerca de llegar a ser una alarma.

\paragraph{}
\textbf{Evolución de las correlaciones por semana:}
\paragraph{}
\irudia{template/figs/tablaAtributos.png}{1}{	Correlaciones por semana.}
\paragraph{}
\irudia{template/figs/asistencia.png}{1}{Gráfico de la variación del coeficiente de correlación del atributo asistencia entre semanas.}
\paragraph{}
\irudia{template/figs/atencionA1.png}{1}{Gráfico de la variación del coeficiente de correlación del atributo atención indicado por el alumno entre semanas.}
\paragraph{}
\irudia{template/figs/motivacionA1.png}{1}{Gráfico de la variación del coeficiente de correlación del atributo motivación indicado por el alumno entre semanas.}







