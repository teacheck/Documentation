\chapter{Empresa}

\section{Plan de viabilidad}
\begin{itemize}
\item{\textbf{Plan de marketing y comercialización:}}
\paragraph{}
\textbf{Análisis DAFO:}
\paragraph{}
\textbf{Debilidades:}
\begin{itemize}
\item{Falta de financiación en los primeros años de puesta en marcha de la aplicación.}
\item{Primer sistema en el ámbito geográfico, tiene nulo atractivo para la compra, esto puede suponer rechazo a los primeros usuarios.}
\item{Poca experiencia.}
\item{Limitación del idioma.}
\item{Limitación de flexibilidad en cuanto a variables a analizar.}
\end{itemize}\\
\textbf{Amenazas:}
\begin{itemize}
\item{Competencia nos tiene ventaja en cuanto a innovación del producto y financiación.}
\end{itemize}\\
\textbf{Fortalezas:}
\begin{itemize}
\item{Gran aceptación en caso de éxito en una sola universidad.}
\item{Flujo de datos y paralelización del sistema para universidades de diferentes tamaño.}
\item{Uso de información aportada por el alumno para el análisis de la información (encuestas semanales).}
\item{El servicio de IA encuentra el motivo por el que ha surgido la alarma.}
\end{itemize}\\
\textbf{	Oportunidades:}
\begin{itemize}
\item{Tendencia de investigación en alza de este canal (ordenadores cuánticos).}
\item{Una vez implantado, beneficios exponenciales a medida que el tiempo pase gracias a la inteligencia artificial.}
\end{itemize}\\
\end{itemize}

\paragraph{}
\textbf{Buyer persona}
El cliente ideal de Teacheck sería aquella universidad que disponga ya de un sistema automático de asistencia y además la información relativa a la motivación ya tención de los alumnos almacenada en su propio servicio. Ya que gracias a estas facilidades seríamos capaces de conseguir información de pasados años y alimentar el servicio de inteligencia artificial eficazmente.

Además, si este cliente o universidad dispone de su propio servidor perfectamente estructurado y asegurado, Teacheck podría centrarse en desarrollar su producto y no tener la necesidad de preocuparse por la seguridad ni la eficiencia del sistema en cuanto a su alojamiento.
Por otro lado, cabe mencionar que si existe una metodología de trabajo eficaz tanto en el personal docente como en los alumnos, los datos a recoger harían de la enseñanza automática un servicio aún más eficaz.

\paragraph{}
\begin{itemize}
\item{\textbf{Plan de viabilidad económico y financiero:}}
\end{itemize}\\
\textbf{Factura:}\\
\irudia{template/figs/factura.jpg}{0.6}{Factura primer año Teacheck.}
\textbf{Ingresos:}\\
\irudia{template/figs/ingresos.jpg}{0.7}{Ingresos primer año Teacheck.}
\irudia{template/figs/inversiones.png}{1}{Inversiones primer año Teacheck.}



\section{Definición de la empresa y del servicio}
Teacheck ofrece a universidades una aplicación web con la cual
mediante sus servicios es capaz de realizar un seguimiento de sus
alumnos, gracias a tecnología innovadora como la inteligencia
artificial.  Es una empresa destinada a instituciones o universidades
tanto públicas como concertadas que busquen realizar una
monitorización automática y precisa de su alumnado. Ofrece un servicio
con el objetivo de facilitar un seguimiento en beneficio tanto del
alumnado principalmente como del profesor. Con esto pretendemos
resolver el problema que actualmente existe con el alto porcentaje de
repetidores y suspensos. Creemos que la mayoría de estos casos suceden
por falta de responsabilidad y desconocimiento de cuando el alumno va
mal o su rendimiento no es el adecuado, y gran parte de estos casos
son evitables. Y para dar solución a este problema, el primer paso es
el interés del profesorado en intentar revertir esta situación y
tratar de alertar al alumno de su situación, que, aunque él ya sea
consciente de ello, en la mayoría de los casos, el hecho de recibir un
aviso o consejo de forma adecuada, esto es, teniendo en cuenta cual es
la manera correcta de decir y plantear los problemas, puede hacerle
cambiar de actitud. Por lo tanto, para conseguir esto es necesario el
seguimiento del alumno en cuestión y tener presentes sus notas,
asistencia, motivación y sus entregables entre otros. Pero aquí se nos
plantea otro problema, y es que hacer el seguimiento de un alumno es
fácil, pero no es lo mismo con diez, veinte o treinta de
ellos. Entonces se complican las cosas ya que el profesor o tutor en
cuestión no podrá cumplir con el seguimiento de todos y los avisos no
podrán llegar a tiempo.
\subsection{Organigrama de la empresa}
Como podemos ver en el organigrama de abajo, la empresa está formada
por 4 departamentos o unidades que se basan en:
\begin{itemize}
\item{El producto, esto es,el desarrollo de toda la aplicación en
  general. Podríamos dividir este departamento en diferentes secciones
  o roles como podrían ser: El análisis y diseño, el desarrollo del
  software y el análisis estadístico y la inteligencia artifical del
  producto.}
\item{Monitorizar y gestionar el capital de la empresa así como las
  inversiones.}
\item{La gestión los empleados y lo relacionado con nuevas
  incorporaciones al equipo y su bienestar.}
\item{Por último ,contactar nuevas universidades y promoción de la
  aplicación.}
\end{itemize}
\subparagraph{}
\irudia{template/figs/organigrama-empresa.jpg}{1}{Organigrama de la
  empresa}
\subsection{Ventajas}
Como hemos mencionado anteriormente, gracias al seguimiento
personalizado que ofrece Teacheck, una universidad podrá organizar de
una forma fácil y sencilla toda información relacionada con el
rendimiento de un alumno o clase en general.Así el personal docente y
los alumnos tendrán una forma de detectar el motivo de una posible
degradación en el ritmo de trabajo de un alumno o profesor.  Además,
como ha sido mencionado anteriormente, gracias a este sistema, se
tendrán en cuenta toda la información de todos los alumnos de una
clase para saber si una asignatura en concreto es implantada de forma
correcta.
\subsection{Desventajas}
La participación del alumnado no es necesaria para el correcto
funcionamiento del sistema, pero su eficiencia y exactitud incrementa
exponencialmente si dispone de la información que un alumno puede
llegar a ingresar. Es por esto que, aún siendo opcional, los alumnos
se puedan sentir presionados o en cierta forma controlados por la
universidad a la hora de tener que ingresar datos como puedan ser las
horas de estudio fuera de clase, la asistencia o la motivación en el
aula.  El personal docente del centro deberá recoger tanto las notas
como la asistencia diaria de sus alumnos y calificar cada clase que
imparta, ya que después, las observaciones hechas durante esa clase,
serán claves para saber el estado de sus alumnos, aumentando así las
responsabilidades de un profesor con una asignatura.
\subsection{Elementos diferenciadores}
Teacheck se situa en la categoría de aplicaciones que intentan
prevenir un deterioro en el rendimiento académico de los
alumnos. Dentro de esta categoría ya existe una institución que
realiza este tipo de actividad. Pero Teacheck ofrece algo más que
solamente el análisis de los datos proporcionados por el profesor,
también ofrece la posibilidad de que un alumno pueda proporcionar
datos los cuales permitirá a la aplicación ser más precisa y eficaz.
Ya que de esta manera, los datos introducidos por los profesores nos
dirán en qué está fallando el alumno y los datos proporcionados por el
alumno, cómo solucionarlo.
\subsection{Gestión de activos}
Teacheck es una pequeña empresa compuesta por 4 trabajadores que acaba de entrar en el mundo laboral. La empresa se divide en estas unidades: Unidad de TI, Unidad financiera, Recursos humanos y Marketing. Al ser una empresa de 4 trabajadores, cada uno de ellos es parte de más de una unidad. 

Todos ellos son parte de la unidad de TI ya que son desarrolladores, pero también trabajan en otras unidades. Uno de ellos trabaja en la unidad financiera, dos de ellos en recursos humanos y el último en marketing.
\irudia{template/figs/departamentos.png}{1}{Departamentos y sus
  técnicos}
Al ser una pequeña empresa y bastante nueva no dispone de muchos activos:
\begin{itemize}
\item{\textbf{4 ordenadores portátiles: }}uno para cada empleado.  Los empleados utilizarán los ordenadores tanto para el desarrollo de la aplicación Teacheck como para los demás trabajos como finanzas, marketing o recursos humanos.
\item{\textbf{Servidor: }}al ser nueva en el mundo laboral y no tener financiación, la empresa no dispone de servidores propios. El servidor de la empresa es alquilado.
\item{\textbf{4 teléfonos móviles: }}los empleados tendrán cada uno un teléfono para la comunicación entre ellos.
\item{\textbf{Router: }}la empresa dispone de un router para la conexión a internet.
\irudia{template/figs/activosEmpresa.jpg}{1}{Activos de la empresa}

\end{itemize}
\subsection{Service desk}
La función service desk es la encargada de registrar todas las
incidencias y solicitudes de servicio de nuestra aplicación. Es el
punto único de contacto entre los usuarios y los proveedores de
servicio. Este servicio no solo será proporcionado a los usuarios de
la aplicación, también será utilizado por los técnicos de la empresa.

Este servicio se ofrece mediante diferentes canales como:
\begin{itemize}
\item{Teléfono: }Los usuarios de la aplicación y los técnicos de la
  empresa podrán utilizar el teléfono como forma de comunicación para
  recibir información o dar avisos de problemas o incidencias.
\item{Correo electrónico: }Podrán enviar correos electrónicos con fin
  de recibir información o avisos de incidencias.
\item{Redes sociales: }Solo se utilizará para recibir información.
\end{itemize}
\subsection{Gestión de incidencias}
Teacheck, es una aplicación que será utilizada por muchas personas
dependiendo de la cantidad de alumnos de dicha universidad. La
aplicación estará en constante mantenimiento pero eso no asegura que
puedan surgir nuevas incidencias.

Para ello, definiremos el proceso de gestión de incidencias, que es el
proceso responsable de la gestión del ciclo de vida de las
incidencias. Esto asegura que se restablezca la operación normal de
servicio lo antes posible y se minimice el impacto negativo al negocio
para que se mantenga el nivel de calidad del servicio acordado.
\subsubsection{Proceso de gestión de incidencias}
\begin{itemize}
\item{\textbf{Identificar incidencia:}} La primera actividad de este
  proceso será identificar la incidencia. Es una de las actividades
  claves ya que es muy importante saber de dónde viene esa incidencia
  para después poder arreglarla. Aquí se debatirá si de verdad es una
  incidencia y hay que tratarla o no.
\item {\textbf{Registrar incidencia:}} El segundo paso será registrar
  la incidencia. Esto nos ayudará en un futuro a resolver de manera
  más fácil y rápida una nueva incidencia ya que es posible que esté
  registrada y solo deberemos seguir los pasos que se siguieron la
  anterior vez.
\item{\textbf{Clasificar y priorizar incidencia:}} Una vez registrada
  pasaremos a clasificar y priorizar la incidencia. Estas, deberemos
  clasificarlas por categorías o niveles y priorizar esas categorías
  para ver qué incidencias son las que corren más prisa solucionar.
\item{\textbf{Asignación de incidencia:}} Teniendo en cuenta la
  disponibilidad de los técnicos, se asignará la incidencia al experto
  en ese tipo de incidencias.
\item{\textbf{Investigar y diagnosticar incidencia:}} Una vez asignada
  la incidencia, el técnico investigará acerca de la incidencia para
  saber cuál es el problema y que se debe hacer para solucionarla.
\item{\textbf{Resolver incidencia:}} Solo queda solucionar la
  incidencia una vez que sabemos cual es el problema y como se
  soluciona.
\item{\textbf{Registrar solución y cerrar incidencia:}} Por último,
  registraremos de manera redactada un informe con los pasos seguidos
  para solucionar el problema y cerraremos la incidencia.
\end{itemize}

\section{ProactivaNet}
Todo esto, hay que gestionarlo y monitorizarlo de alguna manera, y para ello, hemos decidido utilizar ProactivaNet. ProactivaNet es un software que permite hacer la gestión de incidencias, peticiones, problemas, cambios, entregas y niveles de servicio en base a ITIL® e ISO 20000. 

En este software se han utilizado diferentes aplicaciones que se explicarán con más detalle a continuación:
\begin{itemize}
\item{\textbf{Gestión de activos: }}Este es el apartado que nos permite gestionar todos los activos que tenemos. En él, definiremos los grupos de trabajo o unidades que hay en la empresa y los técnicos de cada uno. Y por otro lado gestionaremos los 4 pc’s, los 4 móviles, el router y el servidor que tenemos en Teacheck.
\irudia{template/figs/unidades.jpg}{1}{Unidades de la empresa en ProactivaNet}
Estas son la unidades de las que dispone Teacheck aunque hemos añadido un grupo llamado Administración donde estará Teacheck group, en la que todos los técnicos podrán acceder y hacer cambios.
\irudia{template/figs/unidades2.jpg}{1}{Administradores de la empresa en ProactivaNet}
Como hemos mencionado anteriormente tenemos 4 técnicos + administrador (que puede ser cualquiera de ellos):
\irudia{template/figs/tecnicos.jpg}{1}{Técnicos de la empresa en ProactivaNet}
Cada una de las unidades tiene sus técnicos que en nuestro caso quedaría de esta forma:

Una vez que hemos configurado los grupos de trabajo y sus técnicos, debemos añadir a Proactivanet todos los ordenadores de los que disponemos, y a quien está asignado cada uno.
\irudia{template/figs/ordenadores.jpg}{1}{Ordenadores de la empresa en ProactivaNet}
DESKTOP-L34NDF           ->        Lucas Sousa de Freitas\\
DESKTOP-K5B3RTC        ->        Asier De la Natividad Mohacho\\
LAPTOP01                         ->        Unai Agirre Uribarren\\
DESKTOP-NOSN6LD       ->        Jon Fernandez Bedia\\
\\
ProactivaNet permite crear ordenadores pero también da la posibilidad de introducirlos de forma automática dentro de el apartado:  Utilidades → Audita mi pc. Gracias a esto nuestro pc se añadirá directamente a los activos de Teacheck.
\\
Por otro lado debemos meter los dispositivos moviles:
\irudia{template/figs/moviles.jpg}{1}{Dispositivos móviles de la empresa en ProactivaNet}
\\
Y por último metermos el servidor y el router:
\irudia{template/figs/servidor.jpg}{0.7}{Servidor de la empresa en ProactivaNet}
\\
\item{\textbf{Service Desk: }}Una vez que tenemos los grupos y técnicos definidos, y nuestros activos incluidos, procederemos a crear el proceso de gestión de incidencias que hemos definido anteriormente. Para esto, ProactivaNet proporciona una herramienta dentro de la aplicación Service Desk, llamada Secuencias/Workflows, donde podremos crear un nuevo diagrama llamado Gestión de incidencias en el que dibujaremos el flujo que seguirá nuestra aplicación cuando hay una nueva incidencia.
\irudia{template/figs/gestionIncidencia.jpg}{1}{Gestión de incidencias}
\irudia{template/figs/flujo.jpg}{0.8}{Proceso de gestión de incidencias}

\paragraph{SLA}
El acuerdo de nivel de servicio o ANS escrito entre un proveedor de servicio y su cliente con objeto de fijar el nivel acordado para la calidad del servicio que proporciona teacheck se encuentra en el anexo.\nameref{acuerdoNivelServicio}

\end{itemize}