\chapter{Conclusión}
\paragraph{}
A continuación, se detallan las conclusiones obtenidas respecto a los objetivos marcados.
\begin{itemize}
\item{\textbf{Analizar y automatizar:}}\\
Para empezar, el núcleo del producto es el predictor de inteligencia artificial, debido a que hace cumplir dos de los grandes objetivos. Primero, el de automatizar el seguimiento de los alumnos aligerando la carga de trabajo de quien se ocupase del seguimiento, y segundo, el de analizar los datos y realizar un seguimiento preciso y eficaz. Por lo tanto, en caso de que haya algún problema, se detecta a tiempo y a su vez se detectan los detonantes que han llevado a que dicho problema surja.
\item{\textbf{Desplegar información:}}\\
En cuanto a la plataforma, se ha conseguido desarrollar una aplicación web tal y como se detalla en el informe cumpliendo con lo propuesto, el desplegar la información que un profesor necesita ver, tanto alumnos como alarmas. A su vez, que tuviese una encuesta que el alumno pudiese rellenar.
\item{\textbf{El alumno:}}\\
Tal y como se menciona en los objetivos, contar con la información que pueda aportar el alumno es uno de los puntos diferenciadores respecto a otros productos similares al nuestro, y lo hemos conseguido gracias a las encuestas semanales implantadas en la aplicación.
\item{\textbf{El producto:}}\\
Por último, se ha logrado desarrollar un producto escalable, viable para clientes de distinto calibre, y a su vez seguro, haciendo uso tal y como se menciona en el informe, de diferentes técnicas para cumplir con dicho objetivo.
\end{itemize}