\chapter{Futuras líneas de investigación}
\paragraph{}
A lo largo del transcurso tanto del desarrollo de la idea como del producto, han surgido varios conceptos a implementar que no han podido ser llevados a cabo debido al período de tiempo acordado para la finalización del proyecto. Así pues, estas son las futuras líneas de investigación que Teacheck tomaría en caso de continuar con el desarrollo:
\begin{enumerate}
\item En cuanto al apartado de inteligencia artifical del producto, automatizar el clasificador bayesiano ingenuo de tal forma que, el registro de una nueva posibilidad o valor del atributo pueda ser contemplado según el curso avanza. Esto es, añadir, cada semana, los resultados que hayan obtenido los alumnos con alarma en cada unos de los atributos a analizar. En caso de ya existir, añadir un elemento más a la total acumulado y, si no existe, añadir como nueva posibilidad.
\item Añadir nuevas funcionalidades en la página web:
\begin{enumerate}
\item Un resumen de todo un curso en la aplicación, viendo así la evolución de cada atributo a lo largo del año, o incluso las alarmas de un alumno en todo un año. Junto a esta información, un gráfico, por semana, de los coeficientes de correlación en una asignatura.
\item Nuevas opciones de configuración del sistema, por ejemplo: Traducción al Euskera y el Inglés, visualización para personas con daltonismo, tamaño de la letra en toda la aplicación, etc.
\end{enumerate}
\end{enumerate}
