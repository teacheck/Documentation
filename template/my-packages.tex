% Codificación del archivo / fitxategiaren kudeaketa
\usepackage{ucs}
\usepackage[T1]{fontenc}


% ################################################################
% #######     SIZE OF THE PAGES                     ##############
% ################################################################
\usepackage[left=3.5cm, right=2.5cm, top=4.0cm, bottom=3.0cm]{geometry}
% \usepackage[left=1.5cm, right=2.5cm, top=2.0cm, bottom=2.0cm]{geometry}


% ################################################################
% #######     HEADERS                               ##############
% ################################################################
\usepackage{fancyhdr}           % Para cambiar las cabeceras de las pginas

\pagestyle{fancy}
\renewcommand{\chaptermark}[1]{ \markboth{#1}{} }
\renewcommand{\sectionmark}[1]{ \markright{#1}{} }

\fancyhf{}
\fancyhead[LE,RO]{\thepage}
\fancyhead[RE]{\textit{ \nouppercase{\leftmark}} }
\fancyhead[LO]{\textit{ \nouppercase{\rightmark}} }

\fancypagestyle{plain}{ %
  \fancyhf{} % remove everything
  \renewcommand{\headrulewidth}{0pt} % remove lines as well
  \renewcommand{\footrulewidth}{0pt}
}


	% Redefine plain page style
	\fancypagestyle{plain}{
		\fancyhf{}
		\renewcommand{\headrulewidth}{0pt}
		\fancyfoot[LE,RO]{\thepage}
	}

	% Define pagestyle
	\pagestyle{fancy}
	\fancyhf{}
	% \renewcommand{\chaptermark}[1]{\markboth{ \emph{#1}}{}}
	\fancyhead[LO]{}
	\fancyhead[RE]{\leftmark}
	\fancyfoot[LE,RO]{\thepage}

	% Code for creating empty pages
	% No headers on empty pages before new chapter
	% \makeatletter
	% \def\cleardoublepage{\clearpage\if@twoside \ifodd\c@page\else
		% \hbox{}
		% \thispagestyle{plain}
		% \newpage
		% \if@twocolumn\hbox{}\newpage\fi\fi\fi}
	% \makeatother \clearpage{\pagestyle{plain}\cleardoublepage}

	% Otra opción: considerar si funciona
	% this next section (till \makeatother) makes sure that blank pages
	%% are actually completely blank, cause they're not usually
	\makeatletter
	\def\cleardoublepage{\clearpage\if@twoside \ifodd\c@page\else
		\hbox{}
		\vspace*{\fill}
		\thispagestyle{empty}
		\newpage
		\if@twocolumn\hbox{}\newpage\fi\fi\fi}
	\makeatother


	% \pagestyle{fancy}				% use fancyhdr style
	% \setlength{\headheight}{13pt}

	% Limpiar estilo actual
	% \fancyhead{}
	% \fancyfoot{}
	% % or \fancyhf{}

	% \renewcommand{\headrulewidth}{0.4pt}    % Cabecera: subraya la cabecera (fijar en "0pt" si no se desea).
	% \renewcommand{\footrulewidth}{0pt}      % Pié: subraya el pie de página (fijar en "0pt" si no se desea).

	% There are seven letters you need to know before you can define your own header/footer:
	% E: Even page
	% O: Odd page
	% L: Left field
	% C: Center field
	% R: Right field
	% H: Header
	% F: Footer

% 	\fancyhead[CO,CE]{---Draft---}
% 	\fancyfoot[CO,CE]{Confidential}

% 	\fancyfoot[RO, LE] {\thepage}
% 	% or \fancyhf[FRO,FLE]...
% 
% 	\fancyhead[RE]{\nouppercase{\leftmark}}	% Cabecera: incluye información del nivel superior (Capítulo) % a la derecha (R) de las páginas pares (E), evitando escribir % todo en mayúsculas (que sería la opción por defecto).
% 	% or \fancyhf[HRE]...
% 
% 	\fancyhead[LO]{\nouppercase{\rightmark}}% Cabecera: incluyer información del nivel inferior (Sección) % a la izquierda (L) de las páginas impares (O), evitando escribir % todo en mayúsculas (que sería la opción por defecto).

	% \renewcommand{\chaptermark}[1]{%
		% \markboth{\small\slshape\chaptername{} \thechapter: #1}{}
		% }
	% \renewcommand{\sectionmark}[1]{%
		% \markright{\small\slshape\thesection : #1}
		% }

\renewcommand{\chaptermark}[1]{\markboth{#1}{}}
\renewcommand{\sectionmark}[1]{\markright{\thesection\ #1}}


%% which sections are numbered
\setcounter{secnumdepth}{2}

 
% ################################################################
% #######     Bibliografia                          ##############
% ################################################################
% \usepackage{natbib}
% \newcommand{\citenp}[2][ ]{\citeauthor{#2}#1 (\citeyear{#2})}
% \bibpunct{}{}{;}{a}{,}{,~}
% \newcommand{\myetal}{\emph{et~al.}}
% \bibliographystyle{plainnat4}
\bibliographystyle{apalike}


% To insert development comments (todos, corrections...)
\usepackage[textsize=scriptsize,textwidth=2cm]{todonotes}
% How to use: 
% - \todo{comentario/iruzkina} (insert into tex)
% - \todo[inline]{}

% ################################################################
% #######     FONT TYPES                            ##############
% ################################################################
% Charter
%\usepackage[bitstream-charter]{mathdesign}
% 	\renewcommand{\rmdefault}{mdbch} % charter
\DeclareSymbolFont{usualmathcal}{OMS}{cmsy}{m}{n}
\DeclareSymbolFontAlphabet{\mathcal}{usualmathcal}
%\usepackage{charter}
% \renewcommand{\rmdefault}{bch}
% \renewcommand{\bfdefault}{b}

% times erabili beharrean
\usepackage{mathptmx}
\usepackage[scaled=.90]{helvet}

% \renewcommand{\rmdefault}{ppl}
% \usepackage{mathpazo} % palatino
% \linespread{1.05}        % Palatino needs more leading
% \usepackage[bitstream-charter]{mathdesign}
% \usepackage{libertine}

%\usepackage[scaled]{berasans}

%\usepackage[scaled]{beramono}
% \renewcommand{\sfdefault}{fxbf}
	% libertine
% 		\renewcommand{\rmdefault}{fxlj} % Linux libertine 


% Sans serif


% ################################################################
% #######     GRAPHICS                              ##############
% ################################################################
\usepackage{graphicx}
\DeclareGraphicsExtensions{.png,.gif,.jpg,.pdf}
% \graphicspath{./irudiak/}
% \newcommand{\irudia}[3]{%
	% \begin{figure}[htb!]
	% \centering%
	% \includegraphics[width=#2]{#1}
	% \caption{#3}
	% \label{fig:#1}
	% \end{figure}
% }

\usepackage[figuresright]{rotating}

\newcommand{\fitx}[1]{\texttt{#1}}

%%%%%%%%%%%%%%%%%%%%%%%%%%%%%%%%%%%%%%%%%%%%%%%%%%%%%%%%%%%%%%%%
%%%%%%%%%%%  PARRAFOEN ESTILOA    %%%%%%%%%%%%%%%%%%%%%%%%%%%%
\frenchspacing
\widowpenalty=1000

% \titlespacing{\section}{1pc}{0ex plus .1ex minus .2ex}{1pc}
% \titlespacing{\section}{0pt}{*1}{*1}
\setlength{\parindent}{0cm} % anula indentacion de parrafos
\setlength{\parskip}{1.5ex plus 0.5ex minus 0.5ex}   % establece separacion entre parrafos a 8 puntos

\setlength\headheight{15pt}

\usepackage{setspace} % Lerroen arteko espazioa
%\singlespacing
\onehalfspacing
%\doublespacing
%\setstretch{1.1}

% hobeto ``justifika''tzeko
%\usepackage[protrusion=true,expansion=true]{microtype}


%%%%%%%%%%%%%%%%%%%%%%%%%%%%%%%%%%%%%%%%%%%%%%%%%%%%%%%%%%%%%%%%
%%%%%%%%%%% IZENBURUEN ESTILOA   %%%%%%%%%%%%%%%%%%%%%%%%%%%%
\usepackage[sf,outermarks]{titlesec}
% \usepackage[compact]{titlesec}

\titleformat{\chapter}[display]
  {\bfseries\Large}
  {\filleft\Huge\thechapter. \Large\MakeUppercase{\chaptertitlename}}
  {4ex}
  {\titlerule
	\vspace{2ex}%
	\filright}
  [\vspace{2ex}%
   \titlerule]

% ATalen formatua
\renewcommand{\thepart}{\arabic{part}}
\titleformat{\part}[display]
  {\bfseries \Large}
  {\filcenter \Huge\thepart. \Huge\MakeUppercase{\partname}}
  {4ex}
  {%marra
    \vspace{2ex}%
    \filcenter \huge  \filright} %filcenter
  [\vspace{2ex}%
   ]





%usepackage{calc} % para hacer calculos al establecer las medias ej: \textwidth -2px
% \usepackage{sectsty}
% \newcommand{\cabecerasformatosection}[1]{%
	% {\makebox[0.98\linewidth][l]{#1}}
% }
% \newcommand{\cabecerasformatosubsection}[1]{%
	% {\makebox[0.98\linewidth][l]{\textsl{#1}}}
% }
% \newcommand{\cabecerasformatosubsubsection}[1]{%
	% {\framebox[1.1\width][l]{#1}}
% }
% \sectionfont{\cabecerasformatosection}
% \subsectionfont{\cabecerasformatosubsection}
% \subsubsectionfont{\cabecerasformatosubsubsection}
% \sectionfont{\sffamily}
% \subsectionfont{\sffamily\textsl}
% \subsubsectionfont{\sffamily}


\usepackage{appendix}
% \usepackage{glossaries}
% Erabilera 
% http://en.wikibooks.org/wiki/LaTeX/Glossary
% latexmk erabiliz gero, ikusi http://tex.stackexchange.com/questions/1226/how-to-make-latexmk-use-makeglossaries

% Glosario-en eskuliburu zabaldua
% http://osl.ugr.es/CTAN/macros/latex/contrib/glossaries/glossaries-user.html#x1-140002.2



\usepackage{color}  
\usepackage{xcolor}
\usepackage{colortbl}

\definecolor{light-gray}{cmyk}{0,0,0,.3} 
\definecolor{orange}{rgb}{1,0.7,0}
\definecolor{light-brown}{RGB}{184,134,11}

\definecolor{gray90}{gray}{.90}
\definecolor{gray75}{gray}{.75}
\definecolor{gray95}{gray}{.95}

\definecolor{lightgray}{gray}{.8}
\definecolor{lightlightgray}{gray}{.95}

\definecolor{atzekokolorea}{gray}{.97}
\definecolor{atzekokoloreasol}{gray}{.7}
\definecolor{atzekokoloreafitx}{gray}{.97}
\definecolor{atzekokoloreafitx_markoa}{gray}{.65}

\usepackage{textcomp} % XML kodea formateatzeko

\usepackage{listings}

\lstset{
    tabsize=4,
    basicstyle=\scriptsize,
    upquote=true,
    aboveskip={1.5\baselineskip},
    columns=fixed,
    showstringspaces=false,
    extendedchars=true,
    breaklines=true,
    showtabs=false,
    showspaces=false,
    showstringspaces=false,
    identifierstyle=\ttfamily,
    commentstyle=\color[rgb]{0.133,0.545,0.133},
    stringstyle=\color[rgb]{0.627,0.126,0.941}\ttfamily,
    morekeywords={SCORE},keywordstyle=\color{red},
    emph={SCORE,CODE,ID,LEMA,POS},emphstyle=\color{light-brown},
    moreemph={[2]top,num,ENtitle,TERM,WF,SYNSET,ENdesc,ENnarr,EStitle,ESdesc,ESnarr,EXP,DOC,DOCNO,DOCID,HEADLINE,TEXT},emphstyle={[2]\color{blue}}
}



\lstset{ frame=Ltb,
     framerule=0pt,
     aboveskip=0.5cm,
     framextopmargin=3pt,
     framexbottommargin=3pt,
     framexleftmargin=0.4cm,
     framesep=0pt,
     rulesep=.4pt,
     backgroundcolor=\color{gray90},
     rulesepcolor=\color{black},
     %
     stringstyle=\ttfamily,
     showstringspaces = false,
     basicstyle=\small\ttfamily,
     commentstyle=\color{gray45},
     keywordstyle=\bfseries,
     %
     numbers=left,
     numbersep=15pt,
     numberstyle=\tiny,
     numberfirstline = false,
     breaklines=true,
   }
 
\lstnewenvironment{listing}[1][]
   {\lstset{#1}\pagebreak[0]}{\pagebreak[0]}
\lstdefinestyle{consola}
    {
        numbers=none,
        xleftmargin=\parindent,
        xrightmargin=\parindent,
        aboveskip=3mm,
        belowskip=0.01mm,
        basicstyle=\scriptsize\bf\ttfamily,
        backgroundcolor=\color{gray75}
    }
\lstdefinestyle{no_fileconf}
{
    numbers=none,
    xleftmargin=\parindent,
    xrightmargin=\parindent,
    aboveskip=3mm,
    belowskip=0.01mm,
    basicstyle=\footnotesize\ttfamily,
    backgroundcolor=\color{gray90},
}
\lstdefinestyle{fileconf}
{
        xleftmargin=\parindent,
        xrightmargin=\parindent,
        aboveskip=3mm,
        belowskip=0.01mm,
        basicstyle=\footnotesize\ttfamily,
        backgroundcolor=\color{gray95},
}

\lstset{
	float=[*],
	lineskip=0pt,
	inputencoding=utf8x,
	extendedchars=\true,
% 	texcl=true,
    basicstyle=\scriptsize\ttfamily,             % print whole listing small
	backgroundcolor=\color{atzekokolorea},
	framesep=3pt,frame=single,framerule=0.6pt,framexleftmargin=1pt,
	tabsize=4, 
	linewidth=0.98\linewidth,
	xleftmargin=5pt,
	breaklines=true,
	moredelim=[il][\sffamily\scriptsize\slshape\itshape\color{GRISARGIA}]{º},
	moredelim=[is][\bfseries]{ª}{ª},
%     keywordstyle=\color{black}\bfseries,
% 	fontadjust=true,
                                   % underlined bold black keywords
%     identifierstyle=,              % nothing happens
%     commentstyle=\color{white}, 	% white comments
%     stringstyle=\ttfamily,         % typewriter type for strings
%     showstringspaces=false,        % no special string spaces
%     showtabtruee,        % no special string spaces
% 	upquote=true,
	keepspaces=true,
	% showspaces=true,
	% showtabs=true,
	columns=fullflexible
	}

\lstset{
  literate={á}{{\'a}}1
           {é}{{\'e}}1
           {í}{{\'i}}1
           {ó}{{\'o}}1
           {ú}{{\'u}}1
		   {ñ}{{\~{n}}}1
}


% \renewcommand*\thelstnumber{(\the\value{lstnumber})}

% \lstnewenvironment{komandoak}{\lstset{upquote=true,escapechar=}}{}
% ,numbers=left, stepnumber=1, numbersep=5pt
\lstnewenvironment{komandoak}{
	\lstset{
			upquote=true,
			escapeinside={(!}{!)},
% 			escapebegin=\begin{bfseries},escapeend=\end{bfseries},
% 			morecomment=[l]{\#},
% 			commentstyle=\itshape,
			frameround=tttt
				}}{}


\usepackage{longtable}
\usepackage{multirow}
\usepackage{multicol}

\usepackage{tabulary}

\usepackage{amsmath}
\usepackage{url}
\usepackage{bm} % bold maths symbols

\usepackage{paralist} % compactenum...
\usepackage{booktabs} %\tauletan \toprule, \bottomrule...
% \usepackage{algorithmic} % algoritmoen zerrenda lortzeko
% \usepackage{algorithm} % algoritmoen zerrenda lortzeko

% \usepackage{soul} % text highlighting \hl

% \usepackage[Bjornstrup]{fncychap} 
% \ChTitleVar{\raggedleft\LARGE\bfseries}

\usepackage{tocbibind} % hau ez badut jartzen, gaien aurkibidea eta bibliografia ez dira agertzen pdf-ko bookmark-ean

% ################################################################
% #######     HIZKUNTZA / IDIOMA                    ##############
% ################################################################


% Cover translatable text
\ifdefined\documentLanguageBasque
	\providecommand{\upvehu}{Euskal Herriko Unibertsitatea UPV/EHU}
	\providecommand{\gradua}{\masterName}
	\providecommand{\mapizenburua}{Master Amaierako Tesia}
	\providecommand{\informatikafakultatea}{Informatika Fakultatea}
	\providecommand{\abstract}{Laburpena}
	\providecommand{\authorLabel}{Egilea}
\fi
\ifdefined\documentLanguageSpanish
	\providecommand{\upvehu}{Universidad del País Vasco UPV/EHU}
	\providecommand{\mapizenburua}{Tesis fin de Máster}
	\providecommand{\informatikafakultatea}{Facultad de Informática}
	\providecommand{\abstract}{Resumen}
	\providecommand{\authorLabel}{Autor}
\fi
\ifdefined\documentLanguageEnglish
	\providecommand{\authorLabel}{Autores}
\fi

\usepackage[font=small,labelfont=bf]{caption}

% General translatable text and properties
\ifdefined\documentLanguageBasque
	\usepackage[basque]{babel}
	\addto\captionsbasque{
		\renewcommand{\contentsname}{Gaien aurkibidea}
		\renewcommand{\listfigurename}{Irudien aurkibidea}
		\renewcommand{\listtablename}{Taulen aurkibidea}
		%\renewcommand{\listalgorithmname}{Algoritmoen zerrenda}
		\renewcommand{\appendixname}{Eranskina}%
		\renewcommand{\appendixpagename}{Eranskinak}
		\renewcommand{\appendixtocname}{Eranskinak}
		\renewcommand{\bibname}{Bibliografia}
		\renewcommand{\abstractname}{Laburpena}
		%% Hau ez badut jartzen, Irudia eta Taula maiuskulaz jartzen ditu
		\renewcommand{\tablename}{Taula}
		\renewcommand{\figurename}{Irudia}
		% Glosategietarako
		% \renewcommand*{\glossaryname}{Glosategia}%
		% \renewcommand*{\acronymname}{Akronimoa}%
		% \renewcommand*{\entryname}{Notazioa}%
		% \renewcommand*{\descriptionname}{Deskribapena}%
		% \renewcommand*{\symbolname}{Symboloa}%
		% \renewcommand*{\pagelistname}{Orri zerrenda}%
		% \renewcommand*{\glssymbolsgroupname}{Symboloak}%
		% \renewcommand*{\glsnumbersgroupname}{Zenbakiak}%
		% Sections
		\providecommand{\introduction}{Sarrera}
		\providecommand{\conclusions}{Ondorioak}
	}
	%% Captionak euskarazko ordenean
	\DeclareCaptionLabelFormat{euskaraz}{#2\bothIfSecond{\nobreakspace}{#1}}
	\captionsetup{labelformat=euskaraz}
\fi

\ifdefined\documentLanguageSpanish
	\usepackage[spanish]{babel}
	\addto\captionsspanish{
		\renewcommand{\contentsname}{Tabla de contenido}
		\renewcommand{\listfigurename}{Índice de figuras}
		\renewcommand{\listtablename}{Indice de tablas}
		%\renewcommand{\listalgorithmname}{Índice de algoritmos}
		\renewcommand{\appendixname}{Anexo}
		\renewcommand{\appendixpagename}{Anexos}
		\renewcommand{\appendixtocname}{Anexos}
		\renewcommand{\bibname}{Bibliografía}
		\renewcommand{\abstractname}{Resumen}
		% Sections
		\providecommand{\descripcionProducto}{Descripción del producto}
		\providecommand{\mercado}{Mercado}
		\providecommand{\propuestaValor}{Propuesta de valor}
		\providecommand{\actividadesClave}{Actividades y recursos clave}
	}
	
	% tabla de contenido sin numeracion 
	% \renewcommand\contentsname{Tabla de contenido}
	% lista de figuras 
	% \renewcommand\listfigurename{Lista de figuras}
	% \clearpage

	% lista de tablas
	% \renewcommand\listtablename{Lista de tablas}
		% \renewcommand{tablename}{tabla}
\fi

\ifdefined\documentLanguageEnglish
	\usepackage[english]{babel}
	% Sections

\fi

\usepackage[hyperindex,bookmarks,colorlinks=true,citecolor=blue,urlcolor=blue,linkcolor=blue,pdftex,unicode]{hyperref}

\hypersetup{
	pdfauthor = {\egilea},
	pdftitle = {\izenburua},
	pdfkeywords = {\today},
}

% \makeglossaries	% according to manual, in the preamble and after hyperref

% line in order to check if utf-8 is properly configured: áéíóúñ
